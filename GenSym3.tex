\documentclass{ltjsarticle}

%%%%%%%%%%packages%%%%%%%%%%
%% colors and links
\usepackage[svgnames]{xcolor}
\usepackage[colorlinks,citecolor=DarkGreen,linkcolor=Blue,linktocpage,unicode]{hyperref} 

%% equations
%%%% math
\usepackage{amsmath,amsfonts,amssymb,amsthm}
\usepackage{mathtools}
\usepackage{mathrsfs}
\usepackage{bm}
\usepackage{cancel}
\usepackage{dsfont}
%%%% physics
\usepackage{siunitx}
\usepackage{physics}
%% positioning
\usepackage{array}
\usepackage{float}

%% table
\usepackage{booktabs}
\usepackage{multirow}
\usepackage{hhline}
\usepackage{caption}
\captionsetup{format=hang}
\usepackage{subcaption}

%% figure
\usepackage{graphicx}
\usepackage{tikz}
\usepackage{circuitikz}

%% decorations
\usepackage{titlesec}
\usepackage{picture}

%% framing
\usepackage{fancybox}
\usepackage{boites}
\usepackage{tcolorbox}
\tcbuselibrary{skins,theorems,breakable}

%% citation
\usepackage{cite}

%%%%%%%%%%optional settings%%%%%%%%%%

%%%%%図表並列%%%%%
\makeatletter
\newcommand{\figcaption}[1]{\def\@captype{figure}\caption{#1}}
\newcommand{\tblcaption}[1]{\def\@captype{table}\caption{#1}}
\makeatother

%%%%%itemization%%%%%
\renewcommand{\labelenumi}{\theenumi}
\renewcommand{\theenumi}{(\arabic{enumi})} % 箇条書きをローマ数字に

%%%%%%theorem environments%%%%%
\newtheoremstyle{mystyle}%   % スタイル名
    {}%                      % 上部スペース
    {}%                      % 下部スペース
    {\normalfont}%          % 本文フォント
    {}%                      % インデント量
    {\bf}%                  % 見出しフォント
    {.}%                      % 見出し後の句読点
    {\newline}%                     % 見出し後のスペース
    {\underline{\thmname{#1}\thmnumber{#2}\thmnote{(#3)}}}%
    % 見出しの書式 (can be left empty, meaning `normal')
\theoremstyle{mystyle} % スタイルの適用

\newtheorem{theorem}{定理}[section]
\newtheorem{definition}{定義}[section]
\newtheorem{proposition}[definition]{命題}
\newtheorem{corollary}[theorem]{系}
\renewcommand{\proofname}{証明}

\newtheorem{remark}{Remark. }[section]
\newtheorem{axiom}{公理}[section]
\newtheorem{conjecture}{Conjecture. }[section]

%%%%%%mathtools%%%%%
\mathtoolsset{showonlyrefs=true} % 被参照数式のみ数式番号割り振り
\numberwithin{equation}{section}

\makeatletter
\@addtoreset{equation}{section}
\makeatother

%%%%%%framing%%%%%

\newcommand{\lto}{\longrightarrow}
\newcommand{\lmto}{\longmapsto}
\newcommand{\btl}{\blacktriangleleft}
\newcommand{\btr}{\blacktriangleright}

\newcommand{\then}{\Rightarrow}
\newcommand{\incl}{\hookrightarrow}

%文字
\newcommand{\tbf}[1]{\textbf{#1}}

%括弧類
\newcommand{\lr}[1]{\langle{#1}\rangle}
\newcommand{\ler}[1]{\left({#1}\right)}
\newcommand{\blr}[1]{\left\{{#1}\right\}}
\newcommand{\slr}[1]{\left[{#1}\right]}

%積分測度
\newcommand{\pint}[1]{\int \mathcal{D}{#1}}
\newcommand{\moint}[1]{\int \frac{d^4{#1}}{(2\pi)^4}}
\newcommand{\xint}{\int d^4x}

%空間
\newcommand{\re}{\mathbb{R}}
\newcommand{\cpl}{\mathbb{C}}
\newcommand{\zet}{\mathbb{Z}}
\newcommand{\rpr}[1]{\mathbb{R}P^{#1}}
\newcommand{\cpr}[1]{\mathbb{C}P^{#1}}

%物理でよく使う記号
\newcommand{\kt}[1]{|{#1}\rangle}
\newcommand{\br}[1]{\langle{#1}|}
\newcommand{\brkt}[2]{\langle{#1}|{#2}|{#1}\rangle}
%%d次元作用
\newcommand{\sayou}[1]{\int d^{#1}x ~\mathcal{L}}

%矢印
\newcommand{\lto}{\longrightarrow}
\newcommand{\lmto}{\longmapsto}
\newcommand{\btl}{\blacktriangleleft}
\newcommand{\btr}{\blacktriangleright}

\newcommand{\thenarr}{\Rightarrow} %「ならば」の矢印
\newcommand{\incl}{\hookrightarrow} %inclusionの矢印
\newcommand{\ninf}{\xrightarrow{n\to\infty}} %n\to \inftyの矢印

%数学
%%何らかの空間の圏
%\newcommand{\cat}[1]{\boldmath{{#1}}}
%Hom
\newcommand{\hom}[2]{\mathrm{Hom}({#1}, {#2})}

%slash_on_letter
\newcommand{\son}[1]{{\ooalign{\hfil$#1$\hfil\crcr\raise.167ex\hbox{/}}}}
%メモ: tikzsetでtikzコマンドを定義できる


%線分の中間に矢印を描くためのコマンド
%使い方: \draw[black, ->-={.5}{red}] (0,0)--(1,1);
%↑の時, (0,0)--(1,1)に伸びる線分の中間に赤い矢印が描かれる
\tikzset{->-/.style 2 args={
    postaction={decorate},
    decoration={markings, mark=at position #1 with {\arrow[thick, #2]{>}}}
    },
    ->-/.default={0.5}{}
}

\tikzset{-<-/.style 2 args={
    postaction={decorate},
    decoration={markings, mark=at position #1 with {\arrow[thick, #2]{<}}}
    },
    -<-/.default={0.5}{}
}


%3次元で交差する線分を描くコマンド
%上に来る線分を後に書く
%使い方: 
%\draw (0,0)--(1,1);
%\draw[overarc] (0,1)--(1,0);
\tikzset{
    overarc/.style={
        white, double=red, double distance=1.2pt, line width=2.4pt
    }
}


%

\title{Generalized symmetry Day2}
\author{Shuma NAKASHIBA}
\date{\today}

\begin{document}
\maketitle

\setcounter{tocdepth}{2}
\tableofcontents
\newpage
\section{More About Higher-Form Symmetry}
\subsection{Anomaly of Higher-Form Global Symmerty}
\subsection{Spontaneous Symmetry Breaking in Higher-Form}
\subsection{Example: $U(1)$ Higgs model}
%%%%%%%%%%%%%%%%%
%
%%%%%%%%%%%%%%%%%
\newpage
\section{Discrete Gloal Symmetry}
So far, we have only considered systems with continuous symmetry. 
In that case, we can find conserved $(d-p)$-form current, or closed $d$-form explicitly, 
by considering infinitesimal transformations of group elements parametrized by continuous numbers. \\
 However, we cannot do in the same manner for the case of discrete symmetry, 
such as $\zet_2$ or $\zet_N$. 
Indeed, the concept of "infinitesimal transformation" is totally nonsense for discrete symmetry 
since all the elements are "finitely separated"
\footnote{
    More precisely, 
    discrete group is equipped with discrete topology (離散位相) while 
    continuous group can have the structure of metric space (距離空間). 
    Especially, Lie group is equipped with differential structure (微分構造). 
} from identity. 
Then, how can we characterize the symmetry of discrete group? 
Remember that the theory with (generalized) symmetry has symmetry operators which are topological, 
and symmetry action on charged objects known as defect operators is defined. 
For continuous case, we can construct such topological operators explicitly from the conserved current. 
How can we find such topological operators in the case of discrete symmetry? \\\\
%これ, 答えがない可能性ある? 
\subsection{Example From Lattice: Toric Code}
Toric Code is known as the example of $1$-form 
discrete global symmetry: $\zet^{e}_2\times \zet^m_2$. 
The symmetry operator of $\zet^e_2$ is the Wilson loop 
$W[C]=\prod_{i\in C}\sigma^z_i$
 (corresponding to Magnetic string operator along the non-contractible closed loop), 
and that of $\zet^m_2$ is the 't Hooft loop 
$T[\tilde{C}]=\prod_{i\in \tilde{C}}\sigma^x_i$
(corresponding to Electric string operator along the non-contractible closed loop). \\
 It is known that the ground state of toric code Hamiltonian is degenerated by $4$. 
This can be considered as the consequence of 
"$1$-form symmetry breaking of $Z^e_2\times Z^m_2$" 
(note that the order(位数) of $Z^e_2\times Z^m_2$ is 4). 
In fact, for example, the charged objects under $Z^e_2$ 
(i.e. non-contractible closed magnetic loops $W[C]$) 
have a finite expectation value on the ground states: 
\\
\\
 As we have seen before, the electric and magnetic $1$-form $U(1)$ symmetries causes
mixed 't Hooft anomaly when they are gauged together. 
The similar thing happens in toric code, but the symmetry is $1$-form $\zet_2$ instead of $U(1)$. 

\section{Appendices}
\subsection{A Brief Intro to Toric Code}
\noindent
\tiny
(I think this part should be written by Takumi-san)\\
\normalsize
\subsubsection{topological order}
Generally, a system is said to have topological order if
\cite{DXN}
\begin{itemize}
    \item it has (approximately) degenerate ground states $\{\psi_\alpha\}$ 
    all of which are separated by a finite gap from excited states, 
    \item the degeneracy of $\{\psi_\alpha\}$s are dominated by the topology of the spacetime manifold 
    on which the system is embedded, 
    \item and no loacl operator can either distinguish or induce transitions between two different $\psi_\alpha$s, i.e., 
    \begin{align}
        \langle \psi_1 |\mathcal{O}_{\mathrm{local}} |\psi_2\rangle
    \end{align}
    for any $\mathcal{O}_{\mathrm{local}}$ acting on the Hilbert space. 
\end{itemize}
\subsubsection{Kitaev model}
\subsubsection{toric code Hamiltonian and its property}
The \textbf{toric code} model is constructed on 
a 2D lattice (typically square lattice), with periodic boundary condition on both directions 
(i.e. the Hilbert space is on torus $T^2$). 
We put a qubit on every link. 
The Hamiltonian is 
\begin{align}
    H=-\sum_j A_j - \sum_p B_p, 
    \label{TCmodel}
\end{align}
where $j$ and $p$ is the index of site and plaquette respectively, 
$A_{j}\equiv \prod_{l:\mathrm{link~ending~at~j}} Z_l$ is defined for every site $j$, 
and $B_{p}\equiv \prod_{l:\mathrm{bdy~of~p}} X_l$ is for every plaquette $p$. 
$A_j$ and $B_p$ are sometimes called "Gauss law operator" and "flux operator" respectively. \\
 All of those operators $A_j$ and $B_p$ commute with each other since 
they all share an even number (0, 2, or 4) of $X$s and $Z$s 
(so we always have $(-1)^{\mathrm{even}}$ factor). 
Then, we can diagonalize all terms in \label{TCmodel} simultaneously. 
Let's consider states whose eigenvalue of $A_j$ is 1 for every $j$. 
For such states, all vertex should be associated with an even number of $X_l$s ending on it. 
The states satisfying this condition are "closed-string states", 
where all lines of electric flux are closed and have no charges to end on. 


\subsection{離散ゲージ理論}
\noindent
\small
This part is originally intended to put in the preliminary of Day1 material, 
so it is written in Japanese. \\
\normalsize
 物理学では, $\mathbb{Z}_2$対称性や$\mathbb{Z}_{N}$対称性のような離散的な対称性が重要な役割を果たすことがある. 
このような離散的な内部対称性を持つ理論, すなわち離散ゲージ理論について簡単に述べる. 
\subsubsection{時空の単体分割と離散ゲージ理論}
 離散的な変換の場合, 場の各点ごとに異なる変換を施すという事が意味を持たなくなる
(時空座標が連続的で無限個の点からなるのに対して, 変換群の元は離散的で有限個であるため). 
そこで, 離散ゲージ理論では, 場の定義される時空を単体分割して離散的に取り扱う. 
単体分割とは, 微分可能な多様体を, その「基本単位」である単体の和(形式的線型結合)に分割することである. 
$0$次元の単体($0$-単体)は点, $1$次元単体($1$-単体)は線分, $2$次元単体($2$-単体)は三角形, などなど. 
$p$次元の単体は$p$-単体と呼ばれ, これは$(p+1)$個の互いに独立な頂点$(v_0, \cdots, v_p)$を(向きを考慮して)結んで出来る. \\
 連続的な場の理論におけるゲージ変換では, 時空上の各点に$G$の元を割り当てていた. 
ここではその代わりに, 単体分割された時空において, $p$-単体に対して$G$の元を割り当てる写像(=\textbf{$p$-コチェイン})を考えることにする
(ただし, $G$はAbel群). 
このような写像が"コチェイン"と呼ばれる理由は, $p$-コチェインに対して$(p+1)$-コチェインを返す写像
\begin{align}
    \delta c_p
\end{align}
があるためである
($G$係数のコホモロジーを構成できる). 
\subsubsection{より一般の場合}
\begin{align}
    Z=\sum_{\pi: P\to X}e^{-S[\pi]}
\label{PFdiscrete}
\end{align}
つまり, 離散ゲージ理論では, あらゆる主$G$束の同型類にわたる和を取って分配関数を与える. 
\begin{thebibliography}{99}
    \bibitem{TDBSH} T. Daniel Brennan, Sungwoo Hong. 
    Introduction to Generalized Global Symmetries in QFT and Particle Physics. \href{https://arxiv.org/abs/2306.00912}{\texttt{arXiv: 2306.00912}}
    \bibitem{DXN} Dung Xuan Nguyen.  
    Introduction to Toric code: From higher form symmetry perspective. \href{https://pcs.ibs.re.kr/PCS_Workshops/PCS_Asian_Network_School_&_Workshop_Talks_2023_files/Dung_Xuan_Nguyen.pdf}{slide is available here}
\end{thebibliography}
\end{document}