\documentclass{ltjsarticle}

%%%%%%%%%%packages%%%%%%%%%%
%% colors and links
\usepackage[svgnames]{xcolor}
\usepackage[colorlinks,citecolor=DarkGreen,linkcolor=Blue,linktocpage,unicode]{hyperref} 

%% equations
%%%% math
\usepackage{amsmath,amsfonts,amssymb,amsthm}
\usepackage{mathtools}
\usepackage{mathrsfs}
\usepackage{bm}
\usepackage{cancel}
\usepackage{dsfont}
%%%% physics
\usepackage{siunitx}
\usepackage{physics}
%% positioning
\usepackage{array}
\usepackage{float}

%% table
\usepackage{booktabs}
\usepackage{multirow}
\usepackage{hhline}
\usepackage{caption}
\captionsetup{format=hang}
\usepackage{subcaption}

%% figure
\usepackage{graphicx}
\usepackage{tikz}
\usepackage{circuitikz}

%% decorations
\usepackage{titlesec}
\usepackage{picture}

%% framing
\usepackage{fancybox}
\usepackage{boites}
\usepackage{tcolorbox}
\tcbuselibrary{skins,theorems,breakable}

%% citation
\usepackage{cite}

%%%%%%%%%%optional settings%%%%%%%%%%

%%%%%図表並列%%%%%
\makeatletter
\newcommand{\figcaption}[1]{\def\@captype{figure}\caption{#1}}
\newcommand{\tblcaption}[1]{\def\@captype{table}\caption{#1}}
\makeatother

%%%%%itemization%%%%%
\renewcommand{\labelenumi}{\theenumi}
\renewcommand{\theenumi}{(\arabic{enumi})} % 箇条書きをローマ数字に

%%%%%%theorem environments%%%%%
\newtheoremstyle{mystyle}%   % スタイル名
    {}%                      % 上部スペース
    {}%                      % 下部スペース
    {\normalfont}%          % 本文フォント
    {}%                      % インデント量
    {\bf}%                  % 見出しフォント
    {.}%                      % 見出し後の句読点
    {\newline}%                     % 見出し後のスペース
    {\underline{\thmname{#1}\thmnumber{#2}\thmnote{(#3)}}}%
    % 見出しの書式 (can be left empty, meaning `normal')
\theoremstyle{mystyle} % スタイルの適用

\newtheorem{theorem}{定理}[section]
\newtheorem{definition}{定義}[section]
\newtheorem{proposition}[definition]{命題}
\newtheorem{corollary}[theorem]{系}
\renewcommand{\proofname}{証明}

\newtheorem{remark}{Remark. }[section]
\newtheorem{axiom}{公理}[section]
\newtheorem{conjecture}{Conjecture. }[section]

%%%%%%mathtools%%%%%
\mathtoolsset{showonlyrefs=true} % 被参照数式のみ数式番号割り振り
\numberwithin{equation}{section}

\makeatletter
\@addtoreset{equation}{section}
\makeatother

%%%%%%framing%%%%%

\newcommand{\lto}{\longrightarrow}
\newcommand{\lmto}{\longmapsto}
\newcommand{\btl}{\blacktriangleleft}
\newcommand{\btr}{\blacktriangleright}

\newcommand{\then}{\Rightarrow}
\newcommand{\incl}{\hookrightarrow}

%文字
\newcommand{\tbf}[1]{\textbf{#1}}

%括弧類
\newcommand{\lr}[1]{\langle{#1}\rangle}
\newcommand{\ler}[1]{\left({#1}\right)}
\newcommand{\blr}[1]{\left\{{#1}\right\}}
\newcommand{\slr}[1]{\left[{#1}\right]}

%積分測度
\newcommand{\pint}[1]{\int \mathcal{D}{#1}}
\newcommand{\moint}[1]{\int \frac{d^4{#1}}{(2\pi)^4}}
\newcommand{\xint}{\int d^4x}

%空間
\newcommand{\re}{\mathbb{R}}
\newcommand{\cpl}{\mathbb{C}}
\newcommand{\zet}{\mathbb{Z}}
\newcommand{\rpr}[1]{\mathbb{R}P^{#1}}
\newcommand{\cpr}[1]{\mathbb{C}P^{#1}}

%物理でよく使う記号
\newcommand{\kt}[1]{|{#1}\rangle}
\newcommand{\br}[1]{\langle{#1}|}
\newcommand{\brkt}[2]{\langle{#1}|{#2}|{#1}\rangle}
%%d次元作用
\newcommand{\sayou}[1]{\int d^{#1}x ~\mathcal{L}}

%矢印
\newcommand{\lto}{\longrightarrow}
\newcommand{\lmto}{\longmapsto}
\newcommand{\btl}{\blacktriangleleft}
\newcommand{\btr}{\blacktriangleright}

\newcommand{\thenarr}{\Rightarrow} %「ならば」の矢印
\newcommand{\incl}{\hookrightarrow} %inclusionの矢印
\newcommand{\ninf}{\xrightarrow{n\to\infty}} %n\to \inftyの矢印

%数学
%%何らかの空間の圏
\newcommand{\cat}[1]{\boldmath{{#1}}}

%slash_on_letter
\newcommand{\son}[1]{{\ooalign{\hfil$#1$\hfil\crcr\raise.167ex\hbox{/}}}}
%メモ: tikzsetでtikzコマンドを定義できる


%線分の中間に矢印を描くためのコマンド
%使い方: \draw[black, ->-={.5}{red}] (0,0)--(1,1);
%↑の時, (0,0)--(1,1)に伸びる線分の中間に赤い矢印が描かれる
\tikzset{->-/.style 2 args={
    postaction={decorate},
    decoration={markings, mark=at position #1 with {\arrow[thick, #2]{>}}}
    },
    ->-/.default={0.5}{}
}

\tikzset{-<-/.style 2 args={
    postaction={decorate},
    decoration={markings, mark=at position #1 with {\arrow[thick, #2]{<}}}
    },
    -<-/.default={0.5}{}
}


%3次元で交差する線分を描くコマンド
%上に来る線分を後に書く
%使い方: 
%\draw (0,0)--(1,1);
%\draw[overarc] (0,1)--(1,0);
\tikzset{
    overarc/.style={
        white, double=red, double distance=1.2pt, line width=2.4pt
    }
}


%

\title{一般化対称性輪講: 第一回}
\author{Shuma NAKASHIBA}
\date{\today}

\begin{document}
\maketitle

\setcounter{tocdepth}{2}
\tableofcontents

\newpage
\section{対称性(ordinary symmetry)の復習}
\subsection{場の理論における対称性}
\subsubsection{相関関数と経路積分表示}
\textbf{通常の}場の量子論では, 時空の異なる点$x=(x^0, x^j)$ごとに異なる値を取る\textbf{局所的な}場$\phi(x)$が基本変数となり, 物理量は
場の量(あるいはその汎函数)の相関関数
\begin{align}
    \lr{\phi(x_1)\cdots \phi(x_n)}
    =\frac{1}{Z}\pint{\phi}~ e^{-S[\phi]}\phi(x_1)\cdots \phi(x_n)
\end{align}
\begin{align}
    Z=\pint{\phi}~e^{-S[\phi]}
\end{align}
として表される. 経路積分は, 時空上の各点でパラメタづけられた場$\phi$の可能な全ての配位に関する和であり, その測度は時空座標をパラメタとして形式的に
\begin{align}
    \pint{\phi} = \prod_{x^j}\prod_{\tau}d\phi(x^j, \tau)
\end{align}
と書ける. なお, 次節を除き, 基本的に時間方向についてEuclid化した時空を扱う: 
\begin{align}
    t\to \tau=it, ~
    S_E = \int d\tau ~L_E (q, \partial_\tau q) = -\int d\tau ~L(q, i\partial_\tau q)~.
\end{align}
\subsubsection{作用と対称性}
ここではMinkowski時空上で作用と対称性の関係について考える. 
時空が$(d+1)$次元上で定義されている時, 作用はラグランジアン密度$\mathcal{L}$の積分
$S=\int d^{d+1}x~\mathcal{L}(\phi, \partial_\mu \phi)$
で書ける. 今, 場$\phi$の無限小変換
\begin{align}
    \phi(x)\to \phi'(x) = \phi(x) + \epsilon^a(x) G_{a}\phi(x) (=\phi + \delta \phi)
\end{align}
を考える(ここで, $G^a$は無限小変換の生成子
\footnote{場の変換があるリー群$G$の元によって$\phi(x)\to \phi'=g \phi(x)$のように書けるとする. 
いま, リー群の元$g$がリー環の(ベクトル空間としての)基底$G^a$($a$: 基底のラベル)および指数写像$\exp: \mathfrak{g}\to G$を用いて
$g=\exp({\epsilon^a G_a})$と書けるものとすると, 
無限小変換は$\phi\to \phi' = \phi + \epsilon^a G_{a}\phi$の形になる. 
場の変換が有限群の場合, 無限小変換に相当する操作を考える事ができないので, 注意が必要. 
}
). これは, 座標の変換ではなく場そのものの変換であるため, 時空の対称性(=ローレンツ変換に対する対称性)ではなく場の理論の持つ内部対称性を考えているということになる
\footnote{座標変換を伴わない場の量の変換としては, 例えば場の量をその場で位相変換するなどが考えられる. 
一方, 座標の変換$x^\mu\to x^\mu + \delta x^\mu$に伴って場の量が変化する場合もある. 
例えば, 並進変換$x^\mu \to x^\mu - \epsilon^\mu$を施した時, 
場の量$\phi\to \phi'$は並進後の位置において同じ値を返すべきなので$\phi'(x')=\phi(x)$, すなわち
$\phi'(x)=\phi(x'+\epsilon)=\phi(x)+\epsilon^{\mu}\partial_\mu \phi(x)$
と変換される. }. 
このとき作用の受ける変換を$S[\phi]\to S[\phi+\delta \phi] = S[\phi] + \delta S[\phi]$と書くと, 
\begin{align}
    \label{eq: deltaS}
    \delta S &= S[\phi+\delta \phi] - S[\phi]\\
    &=\int d^{d+1}x \slr{\frac{\partial \mathcal{L}}{\partial \phi}\delta \phi 
    + \frac{\partial\mathcal{L}}{\partial(\partial_\mu \phi)}\delta(\partial_\mu \phi)}\\
    &=\int d^{d+1}x \slr{\frac{\partial \mathcal{L}}{\partial \phi}\epsilon^a G_a \phi 
    + \frac{\partial\mathcal{L}}{\partial(\partial_\mu \phi)}\partial_\mu(\epsilon^a G_a \phi)}\\
    &\stackrel{部分積分}{=}\int d^{d+1}x \slr{\frac{\partial \mathcal{L}}{\partial \phi
    }\epsilon^a G_a \phi - \partial_\mu\ler{\frac{\partial\mathcal{L}}{\partial(\partial_\mu \phi)}}\epsilon^a G_a \phi
    +\partial_\mu\ler{\frac{\partial\mathcal{L}}{\partial(\partial_\mu \phi)}\epsilon^a G_a \phi}}\\
    &=\int d^{d+1}x \slr{\blr{\frac{\partial \mathcal{L}}{\partial \phi
    } - \partial_\mu\ler{\frac{\partial\mathcal{L}}{\partial(\partial_\mu \phi)}}}(\epsilon^a G_a \phi)
    +\partial_\mu\ler{\frac{\partial\mathcal{L}}{\partial(\partial_\mu \phi)} G_a \phi}\epsilon^a (x)
    + \ler{\frac{\partial\mathcal{L}}{\partial(\partial_\mu \phi)} G_a \phi}\partial_\mu \epsilon^a (x)}\\
\end{align}
となる. 
一般の無限小変換に対して作用は不変とならないが, 大域的対称性を持つ理論の場合は大域的変換($\epsilon^a(x)=\epsilon^a$(定数)となる変換)に対して作用の変化分$\delta S$は0となる. 
この時, まず$\epsilon^a G_a \phi$で括った項については, オイラー・ラグランジュ方程式
\begin{align}
    \frac{\partial \mathcal{L}}{\partial \phi
    } - \partial_\mu\ler{\frac{\partial\mathcal{L}}{\partial(\partial_\mu \phi)}}=0
\end{align}
を満たす場の配位(古典的な配位)に対して0となる. 
第3項は大域的変換のもとでは消えるので, 大域的対称性からは第2項が$0$になる事が従う. すなわち, 
\begin{align}
    \partial_\mu\ler{\frac{\partial\mathcal{L}}{\partial(\partial_\mu \phi)} G_a \phi} =: \partial_\mu j^\mu_a = 0
\end{align}
なる保存量$j^\mu_a$の存在が導かれる(内部対称性に対するNoetherカレント
\footnote{今は内部対称性のみを考えていたが, 座標変換も含めて考える場合はこのNoetherカレントの表式は少し変化する: 
%余裕があれば
例えば, 時空のローレンツ変換に対する保存カレント(エネルギー・運動量テンソル, 角運動量)はこの形をとる. }
)
. この$j^\mu_a$は変換群$G$の生成子$G_a$のラベルを持つ, すなわち独立な変換の方向の数だけ保存量$j_a$が存在する. \\
 保存カレントが存在する時, その時間成分($\mu=0$成分)の空間積分$Q_a:=\int d^{d}x~j_a^0 $を考えると, その時間変化は
\begin{align}
    \frac{d}{dt}Q_a &= \int d^{d}x~\partial_0 j_a^0 
    = \int d^{d}x~(\partial_0 j^0+\partial_i j_a^i)
    \stackrel{\partial_\mu j_a^\mu = 0}{=}
    -\int d^{d}x~\partial_i j_a^i
\end{align}
となる. 最右辺について, 「空間の無限遠方で場の量が0になる」という境界条件を課すと, 
この積分は実質的に境界のない空間上での積分となるので, Gaussの法則から0になる. 
ゆえ, この$Q$は時間変化しない量であり, 大域的対称性に対する保存電荷と呼ばれる. \\
 ところで, この保存電荷は, 無限小変換の生成子としての意味を持つ. 実際, 
\begin{align}
    \slr{iQ_a, \phi(y)}&=\int d^d x~\slr{\frac{\partial\mathcal{L}}{\partial(\partial_0 \phi(x))} G_a \phi(x), \phi(y)}\\
    &=i\int d^d x~\slr{\frac{\partial\mathcal{L}}{\partial(\partial_0 \phi(x))}, \phi(y)}G_a \phi(x)\\
    &=i\int d^d x~(-i)\delta(y^0-x^0)\delta(\bm{y}-\bm{x})G_a \phi(x^0, \bm{x})\\
    &=G_a\phi(y) (=\delta \phi(y))\\
\end{align}
となっている. ただし, 3つ目の等号で, 同時刻正準交換関係$\slr{\phi(x), \pi(y)}:=\slr{\phi(x), \frac{\partial\mathcal{L}}{\partial(\partial_0 \phi(y))}}=i\delta^{0}(x^0-y^0)\delta(\bm{x}-\bm{y})$
を用いた. 
無限小変換の生成子$Q_a$が与えられれば, (微小とは限らない)パラメータの組$\theta=\{\theta^a\}_{a\in \Lambda}$($\Lambda$: 添字集合)を持つ対称性変換の演算子$U_{\theta^a}$を, 
指数写像$\exp$により$U_{\theta}=e^{i\theta^a Q_a}$として与える事ができる. 
この演算子の作用は, $\theta$で指定される群$G$の元が場の量$\phi$に対して引き起こす内部変換と同じである. 
実際, これは場の演算子に対して変換
$U_{\theta}\phi(x)U^{-1}_{\theta}=R_{\theta}\phi(x)$
を引き起越す事がわかる
\footnote{ここで, $R_{\theta}$: パラメータ$\{\theta_a\}_{a\in \Lambda}$
($\Lambda$: 添字集合)でパラメタづけられた群$G$の元の, 場の量の空間(ヒルベルト空間)上での表現. 
$G$がリー群である場合, (同じ連結成分に属する)各元は連続パラメータによって指定できる. 
なお, $G$が行列群の場合はこの$R$は行列だと思って問題ない. }. 
(Baker-Campbell-Hausdorffの公式から確かめられる), 
一般に, 対称性演算子の作用によって非自明な変換を受ける場の演算子を\textbf{荷電演算子}と呼ぶ. 
理論に大域的対称性がある時, 場の演算子は電荷の大きさが$1$の荷電演算子として働く. \\
 荷電演算子は結合律を満たす: 
$U_{\theta_2}U_{\theta_1}=e^{i\theta^a_2 Q_a}e^{i\theta^b_1 Q_b}=e^{i(\theta^a_1+\theta^a_2) Q_a}
=U_{(\theta_1+\theta_2)}$. 
最右辺は, パラメータ$\{\theta^a_1+\theta^a_2\}_{a\in \Lambda}$で指定される$G$の元に対応する対称性演算子. 
また, $U_{\theta}$に対して$U_{-\theta}$はその逆元である. すなわち, 保存電荷から構成される対称性演算子は群$G$をなす. \\
 ところで, 保存電荷には無限小変換の生成子のラベルがついており, これは
リー群$G$の特定の表現$R$における独立な生成子の数に対応している. 
従って, 例えば$G$として$U(1)$を取ると, 既約な$U(1)$表現は1次元なので保存電荷は1種類となり, 
また$G$として$SU(3)$を取ると, 場の量が$SU(3)$の基本表現をなす(基本表現の表現空間を張る)時, 
保存電荷の種類は3となる(クォークを記述するQCDはまさにこのような例). \\
 以上をまとめると, 大域的な対称性(大域的変換に対する作用の不変性)がある理論は, 
対称性の観点からは
「保存するベクトルカレント$j^\mu$」, 
「無限小変換の生成子としての保存電荷$Q$」, 
「群$G$の対称性変換を生成する対称性演算子$U_{\theta}$」および
「対称性演算子により非自明な変換を受ける荷電演算子$\phi$」の4つによって特徴づけられると言える. 
\subsubsection{$U(1)$の例}
$U(1)$対称性を持つ$(d+1)$次元の自由フェルミオンの例を考える. 作用は
\begin{align}
    S = \int d^{d+1} x~ \bar{\psi}(\gamma^\mu\partial_\mu + m)\psi
\end{align}
とする($\bar{\psi}:= i\psi^{\dagger}\gamma^0$). 作用から導かれる運動方程式は, 
\begin{align}
    -(\gamma^\mu \partial_\mu + m)\psi = 0, ~
    \partial_\mu(\bar{\psi}\gamma^\mu) -m\psi = 0
\end{align}
となる. 
この作用は, 場の量に対する$U(1)$内部変換
\begin{align}
    \psi\to e^{i\theta}\psi, ~ \psi^\dagger \to e^{-i\theta}\psi^{\dagger}
\end{align}
の下で不変であり, この内部対称性に対するNoetherカレントは
\begin{align}
    j^\mu = \frac{\partial \mathcal{L}}{\partial(\partial_\mu \psi)}i\psi = i\bar{\psi}\gamma^\mu \psi
\end{align}
となる. これは確かに保存則を満たす: 
\begin{align}
    \partial_\mu j^{\mu}
    =i\partial_\mu(\bar{\psi}\gamma^\mu )\psi + i\bar{\psi}(\gamma^\mu\partial_{\mu}\psi)
    \stackrel{\mathrm{EoM}}{=}im\bar{\psi}\psi - im\bar{\psi}\psi = 0~.
\end{align}
保存するカレントに対して, その時間成分の空間積分$Q=\int d^d x j^{0}(x)$は, 無限遠でカレントがゼロになる境界条件が課されている時, あるいは
時間一定超曲面が境界を持たない時, 時間に依存しない保存量となる. 
この時, $Q$は無限小変換の生成子として働く: $[iQ, \psi]=i\psi$. この時, 
$Q$から定義されるユニタリー演算子$U_\theta = e^{i\theta Q}$は, $\psi$に対して$U(1)$の元$g=e^{i\theta}$による有限変換と同じ変換を引き起こす: 
$U_{\theta}\psi U^{-1}_{\theta} = R_\theta \psi = e^{i\theta}\psi$. 
この$U_\theta$たちが群$U(1)$を生成することは, 容易に確認できる. 
%この辺はNamekiの動画を例に
%\subsubsection{相関関数と対称性}
%以上は古典的な(量子化する前の)作用の対称性であったが, 経路積分量子化の下では運動方程式を満たす経路(=古典解)以外からの寄与も取り入れる必要がある. 
%すなわち, 経路積分量子化のもとでは, 作用そのものではなく相関関数の対称性(不変性)を考えなければならない. 
\section{「対称性」の概念の一般化}
以下では, 場の理論の持つ「対称性」の概念を一般化した枠組みについて議論する. 
一般化の方向性としては主に2つある: 
\begin{itemize}
    \item (1)対称性を「高次元化」する: 荷電演算子を点演算子から$p$次元に広がる演算子へと拡張する
    \item (2)対称性の「群構造を弱める」: 逆元が存在しないような変換に対する不変性を, ある種の「対称性」とみなす
\end{itemize}
(1)の方向性により得られる場の理論の構造は\textbf{higher-form symmetry}と呼ばれ, 
(2)の方向性により得られる構造は\textbf{non-invertible symmetry}と呼ばれる. 
\subsection{対称性を特徴付けるもの}
先の例で見たように, 対称性を持つ場の理論には次のような特徴がある. 
\begin{itemize}
    \item (1)保存するカレント$j_\mu$がある(すなわち, $\partial_\mu j^{\mu}=0$)
    \item (2)$j^\mu$の空間積分(電荷)$Q=\int d^{d}x~j_{0}(x)$があり, これは時間変化しない量である
    \item (3)対称性演算子$U_g=e^{i\alpha Q}$が存在し, 局所的な場$\varphi$(charged operator)に対する変換を引き起こす: 
    \begin{align}
        U_g \varphi(x) = R_g\varphi(x)
    \end{align}
\end{itemize}
いま, これを次のように言い換えよう. 
\begin{itemize}
    \item (1')保存する1-form $j=j_{\mu}dx^{\mu}$がある(すなわち, $d\star j=0$)
    \item (2')$j$のd次元部分空間$M^{d}$上の積分量$Q=\int_{M^d}\star j$があり, これは$M^d\to M^d + \partial N^{d+1}$なる"topological deformation"に対して不変である
    \item (3')'$d$次元超平面上に"topological operator"$U_g(M^d)=e^{i\alpha Q}$が存在し, 超平面とリンクを持つ"charged operator"$\varphi_q$に対して変換を引き起こす. 例えば, $U(1)$対称性の場合は次の通り: 
    \begin{align}
        U_g(M^d) \varphi_q(x) = e^{i\alpha q \mathrm{Link}(M^d, x)}\varphi(x)
    \end{align}
\end{itemize}
これを「$0$次対称性($0$-form symmetry)」の定義として位置付けることにすると, 
その「高次元化」として, 場の理論に対する"$p$-form symmetry"を, 次の性質を満たすものとして特徴付ける事ができる. 
\begin{itemize}
    \item (I)保存する$(p+1)$-form $j$がある
    \item (II)$j$の$(d-p)$次元部分空間$\Sigma^{d-p}$上の積分量$Q=\int_{\Sigma^{d-p}}\star j$があり, これは$M^{d-p}\to M^{d-p} + \partial N^{d-p+1}$なる"topological deformation"に対して不変である
    \item (III)$(d-p)$次元トポロジカル演算子$U_g(\Sigma^{d-p})=e^{i\alpha Q}$が存在し, $p$次元の荷電演算子$\mathcal{O}_q(\Sigma^p)$に対する変換を引き起こす. $U(1)$対称性の場合は次の通り: 
    \begin{align}
        U_g(\Sigma^{d-p}) \mathcal{O}(\Sigma^p) = e^{i\alpha q \mathrm{Link}(\Sigma^{d-p}, \Sigma^p)}\varphi(\Sigma^p)
    \end{align}
\end{itemize}
以下では, (1')(2')(3')の意味について説明する. 
\subsubsection{re: 対称性と保存則}
まず, カレントの保存則$\partial_\mu j^{\mu}=0$について再考する. 
この保存カレントは座標の添字を1つ持つベクトル量であるので, ここからその共変ベクトル, あるいは1-form 
$j=j_\mu dx^\mu$を考える事ができる. 
いま, この1-formに対して, その"Hodge双対"
\begin{align}
    \star: \Omega^{1}(M^{d+1})\to \Omega^{d}(M^{d+1}), ~j\mapsto \star j = \frac{\sqrt{|\det{g}|}}{d!}\epsilon_{\mu \mu_1\cdots \mu_d}j^{\mu}dx^{\mu_1}\wedge \cdots \wedge dx^{\mu_d}
\end{align}
を考えると(Minkowski時空の場合$|\det{g}|=1$), 
カレント保存則$\partial_\mu j^\mu=0$が成り立つ時, この$d$-form$\star j$は閉形式になる: 
\begin{align}
    d\star j = \frac{\sqrt{1}}{d!}\frac{\partial j^\mu}{\partial x^\mu}dx^{1}\wedge \cdots \wedge dx^{d+1}=0~
    (\because \partial_\mu j^\mu = 0). 
\end{align}
すなわち, 保存カレント$j^\mu$の存在は, 閉$d$-form$\star j$の存在と換言できる. \\
 次に, 保存電荷$Q$の意味について考える. 
$d+1$次元時空における時刻一定面を$M^{d}_t$と書くと, 保存電荷は
それが定義される積分領域を明示した形で
$Q(M^d_t)=\int_{M^d_t} j^{0}dx^{1}\cdots dx^{d}$と書ける. 
電荷保存則とはこの$Q$が時間に依存しないという事だった. すなわち, 
\begin{align}
    Q(M^d_{t'})-Q(M^d_t)
    =\int_{M^d_{t'}-M^d_t} j^{0}dx^{1}\cdots dx^{d} =0
\end{align}
となる. これは, 演算子$U_{\theta}(M^d_t)=e^{i\theta Q(M^d_t)}$の期待値が
時刻一定面を変更させても不変であることを示している. 
この関係を, (1+1)次元時空上で図式的に表すと, \\
\begin{figure}[!ht]
    \centering
    \begin{circuitikz}
        \tikzstyle{every node}=[font=\LARGE]
        \draw  (0,12.5) rectangle (3.75,8.75);
        \draw [->, >=Stealth] (-0.5,8.25) -- (-0.5,9.5);
        \draw [->, >=Stealth] (-0.5,8.25) -- (0.75,8.25);
        \node [font=\LARGE] at (1,8.25) {$x$};
        \node [font=\LARGE] at (-0.5,9.75) {$t$};
        \draw [ color={rgb,255:red,4; green,51; blue,255}, line width=0.9pt, short] (0,11.25) .. controls (1.75,11.25) and (2,11.25) .. (3.75,11.25);
        \node [font=\Large] at (1.75,11.75) {$U_\theta(M^d)$};
        \node [font=\LARGE] at (-0.25,11.25) {$t'$};
        \draw  (-6.75,12.5) rectangle (-3,8.75);
        \draw [->, >=Stealth] (-7.25,8.25) -- (-7.25,9.5);
        \draw [->, >=Stealth] (-7.25,8.25) -- (-6,8.25);
        \node [font=\LARGE] at (-5.75,8.25) {$x$};
        \node [font=\LARGE] at (-7.25,9.75) {$t$};
        \draw [ color={rgb,255:red,4; green,51; blue,255}, line width=0.9pt, short] (-6.75,10.5) .. controls (-5,10.5) and (-4.75,10.5) .. (-3,10.5);
        \node [font=\Large] at (-5,11.25) {$U_\theta(M_{t}^d)$};
        \node [font=\LARGE] at (-7,10.75) {$t$};
        \node [font=\LARGE] at (-1.75,10.5) {$=$};
        \draw [ color={rgb,255:red,4; green,51; blue,255}, ->, >=Stealth] (-6.75,10.5) -- (-4.75,10.5);
        \draw [ color={rgb,255:red,4; green,51; blue,255}, ->, >=Stealth] (0,11.25) -- (2,11.25);
        \end{circuitikz}
    \label{fig:my_label}
    \end{figure}
\\のようになる. あるいは, 関係式$U_\theta(M^d_{t'})U^{-1}_{\theta}(M^d_t)=1$を図式的に表して, 次のように理解しても良い. \\
\begin{figure}[!ht]
    \centering
    \resizebox{1\textwidth}{!}{%
    \begin{circuitikz}
    \tikzstyle{every node}=[font=\LARGE]
    \draw  (-6.75,12.5) rectangle (-3,8.75);
    \draw  (-0.75,12.5) rectangle (3,8.75);
    \draw [ color={rgb,255:red,4; green,51; blue,255}, line width=0.9pt, short] (-6.75,11.25) .. controls (-5,11.25) and (-4.75,11.25) .. (-3,11.25);
    \node [font=\Large] at (-5,11.75) {$U_\theta(M_{t'}^d)$};
    \node [font=\LARGE] at (-7,11.5) {$t'$};
    \draw [ color={rgb,255:red,4; green,51; blue,255}, line width=0.9pt, short] (-6.75,10.5) .. controls (-5,10.5) and (-4.75,10.5) .. (-3,10.5);
    \node [font=\Large] at (-5,10) {$U_\theta(M_{t}^d)$};
    \node [font=\LARGE] at (-7,10.5) {$t$};
    
    \node [font=\LARGE] at (-2,10.75) {$=$};
    \draw [ color={rgb,255:red,4; green,51; blue,255}, ->, >=Stealth] (-3,10.5) -- (-5,10.5);
    \draw [ color={rgb,255:red,4; green,51; blue,255}, ->, >=Stealth] (-6.75,11.25) -- (-4.75,11.25);
    \draw  (5,12.5) rectangle (8.75,8.75);
    \node [font=\LARGE] at (4,10.75) {$=$};
    \node [font=\Large] at (1,11.75) {$U_\theta(M_{t'}^d - M_t^d)$};
    \draw [ color={rgb,255:red,4; green,51; blue,255}, line width=0.9pt, short] (-0.75,11) .. controls (1,11) and (1.25,11) .. (3,11);
    \draw [ color={rgb,255:red,4; green,51; blue,255}, ->, >=Stealth] (-0.75,11) -- (1.25,11);
    \draw [ color={rgb,255:red,4; green,51; blue,255}, line width=0.9pt, short] (-0.75,10.75) .. controls (1,10.75) and (1.25,10.75) .. (3,10.75);
    \draw [ color={rgb,255:red,4; green,51; blue,255}, ->, >=Stealth] (3,10.75) -- (1,10.75);
    \end{circuitikz}
    }%
    \label{fig:my_label}
    \end{figure}
    最右辺は, 時空上に非自明な演算子が何もないこと, すなわち恒等演算子$\hat{1}$の真空期待値を表している. 
    \subsubsection{トポロジカル演算子}
     さて, 今までは保存電荷は時空の時間一定面上の積分として定義されていたが, 
    これを一般化して, 時空上の任意の$d$次元の(余次元1の)閉じた超曲面$\Sigma^d$上での積分にしてみよう: 
    \begin{align}
        Q=\int_{\Sigma^d} \star j
        =\int \vec{n}_d \cdot \star j 
    \end{align}
    ただし, $\vec{n}_d$は超曲面$\Sigma^d$の法線ベクトルである. 
    $\vec{n}_d$を時間一定面と垂直に取ると, 従来の電荷の定義$\int j^{0}$に一致する. 
    対応して, 対称性演算子も時空上の$d$次元閉超曲面上で定義する: 
    $U_\theta(\Sigma^d)=e^{i\theta Q(\Sigma^d)}$. 
    ここで, $\star j$が閉形式であることから, 適当な$(d+1)$次元の部分時空の境界$\partial N$の上での積分は消える: 
    \begin{align}
        \int_{\partial N} \star j = \int_{N} d\star j = 0. ~ (\because \mathrm{Stokes'~theorem})
    \end{align}
ゆえ, 対称性演算子$U_\theta(\Sigma)$は, 閉曲面$\Sigma^d$に「任意の境界$\partial N^{d+1}$を付け加える」ような変形に対して不変である
\footnote{つまり, 対称性演算子は閉曲面そのものではなく, その同値類に依存する. 
ここで, 同値関係は, 2つの(向きづけられた)閉曲面の差が, 何らかの1次元高い空間の境界として書けるような場合に2者を同一視する
ものとする: $M^d \sim M^{d'}$ where $M^{d'}-M^d = \partial N^{d+1}$. 
これは, (同じラベルを持つ)対称性演算子が時空$X$の$d$次のホモロジー群: $H_{d}(X)=\mathrm{ker}~\partial_d/\mathrm{Im}~\partial_{d+1}$
の異なる元の数だけ区別されるということを意味する
(そのため, 時空が非自明なトポロジーを持っている時は, 異なるホモロジー群の元でラベルされる(=区別される)あらゆる超閉曲面を考える必要がある. 
例えば, $d=1+1$で時空をトーラス$T^2$に取る時, 
$d=1$次元の閉曲面=1次元閉ループには本質的に4種類存在する: 
(1)可縮なもの, (2)穴を囲うもの, (3)側面を囲うもの, (4)(2)と(3)の形式和. 
より, 互いに連続変形で移り変われないトポロジカル演算子の配位が4通り存在する. ). 
なお, これ以降基本的に時空には三角形分割が入っているものとする. \\
 例えば, 時刻一定面は先の同値関係の下で全て同値なので, 
時刻一定面上で定義された対称性演算子は時間変化の下で同じ. 
}
境界を付け加える操作は閉曲面の連続変形(トポロジーを変えない変形)であるため, 
対称性演算子は閉曲面の連続変形の下で不変であるという意味で\textbf{トポロジカル演算子}と呼ばれる性質を持つ. 
図で表すと次の通り. \\
\begin{figure}[H]
    \centering
    \begin{circuitikz}
    \tikzstyle{every node}=[font=\LARGE]
    \draw  (1.5,-6.75) rectangle (5.25,-10.5);
    \draw [ color={rgb,255:red,4; green,51; blue,255}, line width=0.9pt, short] (1.5,-8.75) .. controls (3.25,-8.75) and (3.5,-8.75) .. (5.25,-8.75);
    \draw [ color={rgb,255:red,4; green,51; blue,255}, ->, >=Stealth] (1.5,-8.75) -- (3.5,-8.75);
    \draw  (-5,-6.75) rectangle (-1.25,-10.5);
    \draw [ color={rgb,255:red,4; green,51; blue,255}, line width=0.9pt, short] (-5,-8.75) .. controls (-3.25,-8.75) and (-3,-8.75) .. (-1.25,-8.75);
    \draw [ color={rgb,255:red,255; green,38; blue,0}, line width=0.9pt, short] (2.75,-8.5) .. controls (3.5,-7.25) and (3.75,-7.75) .. (4.25,-8.5);
    \draw [ color={rgb,255:red,255; green,38; blue,0}, line width=0.9pt, short] (2.75,-8.5) -- (4.25,-8.5);
    \draw [ color={rgb,255:red,255; green,38; blue,0}, line width=0.9pt, ->, >=Stealth] (3.5,-8.5) -- (3.25,-8.5);
    \draw [ color={rgb,255:red,255; green,38; blue,0}, line width=0.9pt, ->, >=Stealth] (3.5,-7.75) .. controls (3.75,-7.75) and (3.5,-7.75) .. (3.75,-7.75) ;
    \draw [ color={rgb,255:red,4; green,51; blue,255}, ->, >=Stealth] (-5,-8.75) -- (-3,-8.75);
    \node [font=\LARGE] at (0,-8.75) {$=$};
    \node [font=\large] at (-3,-8.3) {$U_\theta(M_{t}^d)$};
    \node [font=\large] at (3.5,-9.1) {$U_\theta(M_{t}^d)$};
    \node [font=\large] at (3.5,-7.5) {$U_\theta(\partial N^{d+1})$};
    \node [font=\large] at (9.6,-9.1) {$U_\theta(M_{t}^d+\partial N^{d+1})$};
    \draw  (7.75,-6.75) rectangle (11.5,-10.5);
    \draw [ color={rgb,255:red,4; green,51; blue,255}, line width=0.9pt, short] (7.75,-8.75) .. controls (9.5,-8.75) and (9.75,-8.75) .. (11.5,-8.75);
    \draw [ color={rgb,255:red,255; green,38; blue,0}, line width=0.9pt, short] (9,-8.75) .. controls (9.75,-7.5) and (10,-8) .. (10.5,-8.75);
    \draw [ color={rgb,255:red,255; green,38; blue,0}, line width=0.9pt, ->, >=Stealth] (9.75,-8) .. controls (10,-8) and (9.75,-8) .. (10,-8) ;
    \node [font=\LARGE] at (6.5,-8.75) {$=$};
    \draw [ color={rgb,255:red,255; green,255; blue,255}, line width=0.9pt, short] (9,-8.75) -- (10.5,-8.75);
    \end{circuitikz}
    \label{fig:my_label}
    \end{figure}
この図式は, 対称性演算子$U_{\theta}(\Sigma^d)$が, 
閉曲面に対する任意の連続変形のもとでその期待値を不変に保つことを示している
(時空$\Box$の中に演算子の定義域を線分(高次元なら曲面)で示したこのような図式の間の等式は, 
その演算子の真空期待値(あるいは, 演算子によって"twist"されたヒルベルト空間の分配関数)に関する等式と解釈すべきである)
. \\
 $d+1$次元時空内の$d$次元閉超曲面上で定義された対称性演算子
$U_{\theta}(\Sigma^d)$に対して, 
荷電演算子$\mathcal{O}(x)$は$0$次元の点演算子である. 
$\mathcal{O}=\phi$として, $U_{\theta}(\Sigma^d)$の$\phi(x)$に対するある時刻での作用: 
$U_{\theta}(M_t^d)\phi(x) U^{-1}_{M_t^d} = R_{\theta}\phi(x)$を図示すると, 
$d+1=2, 3$の場合のそれぞれについて次のようになる. 
\begin{figure}[H]
    \centering
    \resizebox{1\textwidth}{!}{%
    \begin{circuitikz}
    \tikzstyle{every node}=[font=\normalsize]
    
    \draw  (1.5,-8.75) rectangle (5.25,-12.5);
    \draw  (-5,-8.75) rectangle (-1.25,-12.5);
    \draw [ color={rgb,255:red,4; green,51; blue,255}, line width=0.9pt, short] (-5,-11) .. controls (-3.25,-11) and (-3,-11) .. (-1.25,-11);
    \draw [ color={rgb,255:red,4; green,51; blue,255}, ->, >=Stealth] (-1.25,-11) -- (-3.25,-11);
    \node [font=\LARGE] at (0,-10.75) {$=$};
    \draw  (7.75,-8.75) rectangle (11.5,-12.5);
    \node [font=\LARGE] at (6.5,-10.75) {$=$};
    \draw [ color={rgb,255:red,4; green,51; blue,255}, line width=0.9pt, short] (-5,-10.5) .. controls (-3.25,-10.5) and (-3,-10.5) .. (-1.25,-10.5);
    \node [font=\normalsize] at (-3.25,-11.5) {$U^{-1}_{\theta}(M_t^d)$};
    \node [font=\normalsize] at (-3.25,-10) {$U_{\theta}(M_{t+\Delta t}^d)=U_{\theta}(M_{t}^d)$};
    \draw [ color={rgb,255:red,4; green,51; blue,255}, ->, >=Stealth] (-5,-10.5) -- (-3,-10.5);
    \node [font=\normalsize] at (-5.5,-11) {$t$};
    \node [font=\normalsize] at (-5.5,-10.5) {$t+\Delta t$};
    \draw [ line width=0.9pt ] (2,-7.5) circle (0cm);
    \node [font=\normalsize] at (-3.25,-10.75) {$\bullet$};
    \node [font=\normalsize] at (-2.1,-10.75) {$\phi(x)$};
    \node [font=\normalsize] at (3.25,-10.75) {$\bullet$};
    \node [font=\normalsize] at (3.75,-10.75) {$\phi(x)$};
    \draw [ color={rgb,255:red,4; green,51; blue,255} , line width=0.9pt ] (3.25,-10.75) circle (0.25cm);
    \node [font=\normalsize, color={rgb,255:red,4; green,51; blue,255}] at (9.5,-10.75) {$\bullet$};
    \node [font=\normalsize] at (10.3,-10.75) {$R_{\theta}\phi(x)$};
    \end{circuitikz}
    }
    \caption{$d=1+1$次元時空における, 対称性演算子$U_{\theta}$の荷電演算子$\phi$に対する作用. }
    \label{fig:my_label}
    \end{figure}
    \begin{figure}[H]
        \centering
        \resizebox{1\textwidth}{!}{%
        \begin{circuitikz}
        \tikzstyle{every node}=[font=\LARGE]
        \draw [ line width=0.9pt ] (-2,-17.75) -- (6.25,-17.75) -- (10.25,-14.75) -- (2,-14.75) -- cycle;
        \draw [ line width=0.9pt ] (19.25,-11) circle (0cm);
        \draw [ line width=0.9pt ] (17.75,-14.5) circle (3.25cm);
        \draw [line width=0.9pt, ->, >=Stealth] (-2.5,-19.25) -- (-2.5,-16.75);
        \draw [line width=0.9pt, ->, >=Stealth] (-2.5,-19.25) -- (0,-19.25);
        \draw [line width=0.9pt, ->, >=Stealth] (-2.5,-19.25) -- (-0.25,-17.75);
        \node [font=\large, rotate around={-90:(0,0)}] at (4,-15.5) {$\bullet$};
        \draw [ line width=0.9pt ] (-1.5,-15.25) -- (6.75,-15.25) -- (10.75,-12.25) -- (2.5,-12.25) -- cycle;
        \draw [line width=0.9pt, ->, >=Stealth] (0,-15.25) -- (3,-15.25);
        \draw [ line width=0.9pt ] (17.75,-14.25) ellipse (3.25cm and 0.5cm);
        \draw [ line width=0.9pt ] (17.75,-14.75) ellipse (3.25cm and 0.5cm);
        \draw [line width=0.9pt, ->, >=Stealth] (17.5,-14.75) -- (18,-14.75);
        \draw [line width=0.9pt, ->, >=Stealth] (18,-13.75) -- (17.75,-13.75);
        \draw [line width=0.9pt, ->, >=Stealth] (18.25,-15.25) -- (17.75,-15.25);
        \draw [line width=0.9pt, ->, >=Stealth] (17.5,-14.25) -- (18,-14.25);
        \draw [ color={rgb,255:red,255; green,255; blue,255} , line width=0.9pt ] (21,-14.25) rectangle (21.25,-14.75);
        \draw [ line width=0.9pt ] (29.5,-11.25) circle (0cm);
        \draw [ color={rgb,255:red,255; green,255; blue,255} , line width=0.9pt ] (31.25,-14.5) rectangle (31.5,-15);
        \draw [ color={rgb,255:red,255; green,255; blue,255} , line width=0.9pt ] (14.25,-14.25) rectangle (14.5,-14.75);
        \draw [ line width=0.9pt ] (28.25,-14.5) circle (3.25cm);
        \node [font=\large] at (17.75,-14.5) {$\bullet$};
        \node [font=\normalsize] at (28.25,-14.5) {$\bullet$};
        \draw [ line width=0.9pt , dashed] (28.25,-14.5) ellipse (3.25cm and 0.75cm);
        \draw [line width=0.9pt, ->, >=Stealth] (7,-12.25) -- (6,-12.25);
        \draw [line width=0.9pt, ->, >=Stealth] (3,-17.75) -- (2.5,-17.75);
        \draw [line width=0.9pt, ->, >=Stealth] (5.25,-14.75) -- (5.5,-14.75);
        \node [font=\large] at (4.75,-13.5) {$U_{\theta}(M_{t+\Delta}^d)$};
        \node [font=\large] at (4.25,-16.25) {$U^{-1}_{\theta}(M_t^d)$};
        \node [font=\LARGE] at (12,-14.75) {$=$};
        \node [font=\LARGE] at (23,-14.75) {$=$};
        \node [font=\LARGE] at (34.25,-14.75) {$=$};
        \node [font=\LARGE] at (38,-14.75) {$\bullet$};
        \node [font=\LARGE] at (38.5,-14.25) {$R_{\theta}\phi(x)$};
        \node [font=\LARGE] at (5,-15.25) {$\phi(x)$};
        \end{circuitikz}
        }%
        \caption{$d=2+1$次元時空における, 対称性演算子$U_{\theta}$の荷電演算子$\phi$に対する作用. }
        \label{fig:my_label}
        \end{figure}
このように, 対称性演算子の間に荷電演算子が「囲われる」, あるいは適当なtopological deformationによってそのような配位にできる時, 
荷電演算子は対称性演算子によって非自明な作用を受けることになる. 
別の見方として, 対称性演算子が荷電演算子を「乗り越える」際に非自明な変換を受ける, という解釈もできる. すなわち, 図式で描くと
\begin{figure}[H]
    \centering
    \begin{circuitikz}
        \tikzstyle{every node}=[font=\LARGE]
        \draw  (0,12.5) rectangle (3.75,8.75);
        \draw [ color={rgb,255:red,4; green,51; blue,255}, line width=0.9pt, short] (0,10.75) -- (3.75,10.75);
        \draw  (-6.75,12.5) rectangle (-3,8.75);
        \draw [ color={rgb,255:red,4; green,51; blue,255}, line width=0.9pt, short] (-6.75,10.75) -- (-3,10.75);
        \node [font=\large] at (-4.75,11.25) {$U_\theta(M_{t}^d)$};
        \node [font=\large] at (2,10.25) {$U_\theta(M_{t}^d)$};
        \node [font=\LARGE] at (-1.75,10.5) {$=$};
        \draw [ color={rgb,255:red,4; green,51; blue,255}, ->, >=Stealth] (-6.75,10.75) -- (-4.75,10.75);
        \draw [ color={rgb,255:red,4; green,51; blue,255}, ->, >=Stealth] (0,10.75) -- (2,10.75);
        \node [font=\normalsize] at (-4.7,10.25) {$\bullet$};
        \node [font=\normalsize] at (-4,10.25) {$\phi(x)$};
        \node [font=\normalsize] at (2,11.25) {$\bullet$};
        \node [font=\normalsize] at (2.7,11.25) {$R_{\theta}\phi(x')$};
        \end{circuitikz}
    \label{fig:my_label}
    \end{figure}
    \noindent
    という変換になる. 
    これは, 荷電演算子が対称性演算子を何も無しに「すり抜ける」ことは出来ないということを意味する. 
    そういう意味で, 荷電演算子は時空の中の非自明な点=\textbf{defect}をなしていると考えられる. \\
 このような, 対称性演算子に対する荷電演算子の作用を, 経路積分の観点から議論しよう. 
いま, 
閉曲面$\Sigma^d = \partial N^{d+1}$上に対称性演算子$U_{\theta}(\Sigma^d)$が定義されているとする
\footnote{既に見たように, 閉曲面上の$U_{\theta}$による演算子$\mathcal{O}$への作用$U_{\theta}(\Sigma^d)\mathcal{O}$は, 
時間一定面上の$U_{\theta}(M_t^{d})$およびその逆元で挟んだ時の作用$U_{\theta}(M^d_t)\mathcal{O}U^{-1}_{\theta}(M^d_t)$
とトポロジカルに等価}. 
この時, 時空上の場の演算子のうち, $N^{d+1}$に含まれるもののみが$U_{\theta}$による変換を受ける. 
\begin{figure}[H]
    \centering
    \resizebox{0.4\textwidth}{!}{%
    \begin{circuitikz}
    \tikzstyle{every node}=[font=\LARGE]
    \draw [ line width=0.9pt ] (8.25,5.5) rectangle (20.75,-6.75);
    \draw [ color={rgb,255:red,227; green,36; blue,0} , fill={rgb,255:red,168; green,198; blue,254}, line width=1.6pt ] (14.5,-0.75) ellipse (4.75cm and 2.75cm);
    \node [font=\LARGE] at (13,-1.75) {$N^{d+1}$};
    \node [font=\LARGE] at (14.5,-0.75) {$\bullet$};
    \node [font=\LARGE] at (18.25,3.5) {$\bullet$};
    \node [font=\LARGE] at (15.25,-1) {$x_1$};
    \node [font=\LARGE] at (19,3.25) {$x_2$};
    \node [font=\LARGE, color={rgb,255:red,255; green,38; blue,0}] at (16.75,-4) {$M^{d}=\partial N^{d+1}$};
    \end{circuitikz}
    }%
    \caption{時空上の閉曲面上に演算子$U_\theta$がいる時, 水色領域(内側)の荷電演算子のみが対称性変換を受ける. }
    \end{figure}
    \noindent
    簡単のため$U(1)$の場合を考え, 対称性演算子を$e^{i\theta Q(M^d_t)}$のように書く. 
    すると, この演算子による変換は, 時空座標に依存するパラメータ
    $$
        \theta(x) =
        \begin{cases}
          \theta,  & \text{if }~ x \in N^{d+1} \\
          0,  & \text{otherwise }
        \end{cases}
        $$
        による時空全体への変換とみなすことができる. 
        より, このような変換による作用の変化分は, \eqref{eq: deltaS}式を参考にすると
        \begin{align}
            \label{eq: deltaS}
    \delta S 
    &=\int d^{d+1}x 
    \partial_\mu\ler{\frac{\partial\mathcal{L}}{\partial(\partial_\mu \phi)} G_a \phi}\epsilon^a (x)\\
    &=\int_{x\in N^{d+1}} d^{d+1}x \partial_\mu j^{\mu}\theta
        \end{align}
\subsection{Higher-form symmetry}
さて, 今まで見てきたように, 群$G$の(内部)対称性のある$d+1$次元時空上の場の理論には, 
「closed $d$-form
\footnote{$1$-formのHodge双対. }
の保存カレント」
「$d$次元閉超曲面上の保存電荷/ 対称性演算子」
「$0$次元の荷電演算子」
という要素が備わっていた. 
いま, 理論がこのような性質を持つ時, \textbf{$0$-form symmetry}があると呼ぶことにしよう. 
$0$-formの$0$とは, 荷電演算子が$0$次元の点演算子($0$-form, つまりただの関数)であることに由来する. \\\\
 $0$次元の荷電演算子に作用する演算子によって生成される対称性を$0$-form symmetryと呼ぶならば, 
$p$次元の広がりを持つ(=$p$次元超曲面上での$p$-formの積分によって定義される)荷電演算子およびそれに作用する演算子がある時, 
その演算子の生成する「対称性」を, \textbf{$p$-form symmetry}と呼ぶのは自然であろう. \\
$p$-form symmetryを議論する上では, 演算子が定義されるあらゆる超曲面を$d+1$次元時空内への埋め込みと考えて, 
次元の代わりに余次元(codimension)で考えると便利なことがある. 例えば, 
$0$-form symmetryの場合, 対称性演算子は次元$d$の超曲面上で定義されているが, 
これは$d+1$次元時空の中の余次元$1$の超曲面とも言える. 
この自然な拡張として, $p$-form symmetryの場合の対称性演算子は, 
$d+1$次元時空の中の余次元$(d+1)-(p+1)=d-p$なる超曲面上で定義されたものとなる. 
以下に, $0$-form symmetryと$p$-form symmetryの間の関係を示す. \\
\begin{table}[H]
    \centering
    \begin{tabular}{|c|c|c|}
        \hline
         & $0$-form symmetry & $p$-form symmetry \\
        \hline
        保存カレント & $d$-form & $(d-p)$-form \\
        \hline
        保存電荷/対称性演算子 & over codim-$1$ surface& over codim-$(p+1)$ surface\\
        \hline
        荷電演算子 & $0$-dim defect & $p$-dim defect \\
        \hline
    \end{tabular}
    \caption{$0$-form symmetryと$p$-form symmetryの間のアナロジー. }
    \label{tab:label}
\end{table}
なお, ここで$p$-form symmetryを「対称性」と括弧付きで書いたのは, 
通常の意味でのgroup-likeな対称性とは異なるからである. 
例えば, $0$-form symmetry(群の対称性)の場合, 
非可換な群$G$を考えれば, その対称性演算子の作用同士は非可換になり得る. 
一方で, $p$-form symmetryの場合は必ず可換である. 
\subsubsection{例: $(3+1)$次元$U(1)$Maxwell理論}
3+1次元の時空において, 物質場が存在しない「純粋な」Maxwell理論を考えてみよう. 
ゲージ場の配位を記述する作用は, $U(1)$ Yang-Mills作用
\begin{align}
    S = \frac{1}{2g^2} \int F\wedge \star F = -\frac{1}{4g^2}\int F_{\mu\nu}F^{\mu\nu}
\end{align}
である. ここで, $F=dA$は場の強さを表す$2$-form. 
$A$のこれは, $2$-form current$\star F$に$F$を結合させたものとみなすことが出来る. 
\newpage
\section{Preliminaries}
この分野を勉強するにあたって役に立ちそうな知識を, ごくごく簡単にまとめます. 
higher-form symmetryとその例を理解するにはゲージ理論の知識が, およびnon-invertible symmetryの例を理解するにはCFTの知識が多少必要になります. 
また, 一般の多様体上で微分や積分を局所座標に依存しない形で定式化するためには, 微分形式の言葉が必要になります. 
\subsection{微分形式}
ここでは, 時空多様体$M$は微分可能であり, 局所座標系$\{(U, \phi)\}$($\phi: U\to \re^{d}$)が与えられているものとします. 
そのため, あらかじめ局所表示されたものとして微分形式を取り扱います. 
\subsubsection{$1$-formの導入: 2次元平面を例に}
いま, 2次元平面$M=\re^{2}$の上の各点$(x,y)$ごとにベクトル$(a_x(x,y), a_y(x,y))$が定義されている, つまりベクトル場が定まっているとする. 
このベクトル場を, $M$上の曲線$\gamma: [0,1]\to M, ~ t\mapsto (x(t), y(t))$に沿って積分したい. 
\begin{align}
    \begin{tikzpicture}
        \draw (-2,-2)--(2,-2)--(2,2)--(-2,2)--(-2,-2);
        \draw[->] (-1,-1) .. controls (-0.5, -1.2) and (0.5,1.2) .. (1,1);
        \node[above] at (0,0) {(x(t), y(t))};
        \node[above] at (-0.5,-0.5) {\gamma};
        \fill[black](0,0) circle [radius=0.05];
    \end{tikzpicture}
\end{align}
この積分は, 
\begin{align}
    \int_{0}^{1} \left(a_x(x(t), y(t))\frac{dx}{dt}dt + a_y(x(t), y(t))\frac{dy}{dt}dt \right)
    \equiv \int_{\gamma} a_x dx + a_y dy
\end{align}
と書ける. 
点$p=(x,y)\in M$に対して, $\ler{\frac{\partial}{\partial x}}_p : C^{\infty}(M)\to \mathbb{R}, f\to \frac{\partial f}{\partial x}(x,y)$および
$\ler{\frac{\partial}{\partial y}}_p : f\to \frac{\partial f}{\partial y}(x,y)$
という2つの写像を定める時, 基底$\blr{\ler{\frac{\partial}{\partial x}}_p, \ler{\frac{\partial}{\partial y}}_p}$で張られる線型空間$T_pM$は$M$の点$p$における接空間と呼ばれ, 
$T_pM$の元は接ベクトルと呼ばれる. 接ベクトルは時空座標の軌跡の情報を持つ. \\
 いま, 接空間の双対空間$T^{*}_pM$, すなわち$\phi: v\in T_p M \mapsto \phi(v)\in \mathbb{K}$ によって張られる線型空間を考える
(係数体$\mathbb{K}$はなんでも良いが, 以下では$\mathbb{R}$とする). 
$T^{*}_pM$の基底を$\blr{dx_p, dy_p}$と書き, これを
双線型写像$\langle \rangle: T_pM \times T^{*}_pM \to \re$によって
\begin{align}
   & \langle \ler{\frac{\partial}{\partial x}}_p, dx_p \rangle = 1,~~ \langle \ler{\frac{\partial}{\partial y}}_p, dx_p \rangle = 0\\
    &\langle \ler{\frac{\partial}{\partial x}}_p, dx_p \rangle = 1,~~ \langle \ler{\frac{\partial}{\partial y}}_p, dy_p \rangle = 1
\end{align}
となるものとする. この時, $dx$および$dy$を, 「各点$p=(x,y)\in M$ごとに, 余接空間$T_p M$の基底$\blr{dx_p, dy_p}$を与えるもの」として, 
$\omega\in \Omega^{1}(M): M\to T^{*}M$を「各点$p=(x,y)$ごとに余接ベクトル$\omega_x(x, y)dx_p + \omega_y(x,y)dy_p$を与えるもの」とする. 
このような$\omega$を$M$上の$1$-formと呼ぶ. 
集合$\Omega^{1}(M)$は, $dx$および$dy$の$C^{\infty}(M)$係数線型結合($C^{\infty}$-加群)の構造を有する. \\
 $1$-formは, 各点における接ベクトルを与えるごとに数を返すため, 「経路$\gamma$に沿った$1$-formの積分」を定義できる: 
\begin{align}
    \int_{\gamma} \omega
    =\int_{0}^{1} \left(\omega_x(x(t), y(t))\frac{dx}{dt}dt + \omega_y(x(t), y(t))\frac{dy}{dt}dt \right)~.
\end{align}
すなわち, $1$-form$\omega_x(x, y)dx_p + \omega_y(x, y)dy_p$の積分は, 接ベクトル$\omega_x(x, y)\ler{\frac{\partial}{\partial x}}_p + \omega_y(x,y)\ler{\frac{\partial}{\partial y}}_p$の積分と結局同じである. \\
 $dx_p$および$dy_p$を点$p=(x,y)$における線素とみなせば, $\omega_p = \omega_x(x,y)dx_p + \omega_y(x,y)dy_p$を, 単に「軌跡$\gamma$に沿った関数の微小変化」と考えることもできる. 
実際, ある関数$f: M\to \re$を用いて$\omega_x(x,y)=\partial_x f(x,y)$, $\omega_y(x,y) = \partial_y f(x,y)$と書ける時, 
$\omega_p = \partial_x f(x,y) dx_p + \partial_y f(x,y) dy_p$で, これはまさに関数$f$の点$p$における微小変化の形. 
\subsubsection{微分形式の定義}
一般の$n$次元可微分多様体においても同様に, 各点$p\in M$の接空間$T_pM$に対する双対空間(余接空間)を考える事で, 
$1$-form $\omega\in \Omega^{1}(M)$を定義出来る. 
多様体$M$の局所座標表示$\{U, \phi\}$が与えられているとすると, 各点$p\in U$上での$\omega$の局所表示は, 一般に$\omega=\sum_{\mu}f_{\mu}(x)dx^{\mu}\in \Omega^{1}(M)$の形をとる. 
この時, 2次元平面の場合と同様に, $1$-formの経路$\gamma: [0,1]\to M$に沿った積分というものを考える事ができる(「経路$\gamma$に沿った積分」を, $1$-formに対して数を返す対応と見なす事ができる). 
$1$-formは接ベクトルの"dual"であり, 各点$p\in M$における接ベクトル$v_p=\sum_{\mu}v^{\mu}(x)\ler{\frac{\partial
}{\partial x}}_p$に対して$1$-formを割り当てる写像を, 
\begin{align}
    v_p \mapsto \omega_p = \sum_{\mu\nu}g_{\mu\nu}v^{\nu}(x)dx^{\mu}_p
\end{align}
によって定める事ができる. 
以下, 局所表示の存在を前提として, 特定の点$p\in M$への依存性を明示しないものとする. \\
 $1$-form全体の集合$\Omega^{1}(M)$に対して, 各点ごとに二項演算$\wedge$を定める: 
\begin{align}
    \omega\wedge \eta = \ler{\sum_{\mu}\omega_\mu(x)dx^{\mu}}\ler{\sum_{\nu}\omega_\nu(x)dx^\nu}
    =\sum_{\mu, \nu}\omega_{\mu}(x)\eta_{\nu} dx^{\mu}\wedge dx^{\nu}
\end{align}
これは$\Omega^{1}(M)$の基底の部分について反対称な演算である: $dx^{\mu}\wedge dx^\nu = -dx^\nu \wedge dx^\mu$. 
$1$-form同士のwedge積$\omega\wedge \eta$は, $dx^{\mu}\wedge dx^\nu$の$C^{\infty}(M)$係数線型結合($C^{\infty}(M)$-加群)の形をとる. 
ここで, $M$上の"2-form"全体の集合$\Omega^{2}(M)$を, $\omega_2 = \sum_{i_1, i_2}f_{i_1, i_2}(x)dx^{i_1}\wedge dx^{i_2}$なる形の元によって生成される$C^{\infty}(M)$-加群として定義すると, 
wedge積は2つの1-formに対して2-formを返す反対称な演算であると言える. 
2-formは, 各点$p\in M$において接ベクトル空間の直積$T_pM\times T_p M$を数に移す写像を定める. \\
 より一般に, $M$上の"k-form"全体の集合$\Omega^{k}(M)$を, $\omega_k = \sum_{i_1, \cdots, i_n}f_{i_1, \cdots, i_n}(x)dx^{i_1}\wedge \cdots \wedge dx^{i_n}$なる元により生成される$C^{\infty}(M)$-加群として定義する. 
この時, $p$-formと$q$-formの間のwedge積を, $(p+q)$-formを返す反対称な演算として定義出来る: 
$\alpha_p = \sum_{i_1, \cdots, i_p}\alpha_{i_1, \cdots, i_p}(x)dx^{i_1}\wedge \cdots \wedge dx^{i_p}$
および$\beta_q = \sum_{j_1, \cdots, j_q}\beta_{j_1, \cdots, j_q}(x)dx^{j_1}\wedge \cdots \wedge dx^{j_q}$に対し, 
\begin{align}
    &\alpha_p \wedge \beta_q = \sum_{i_1, \cdots, i_p, j_1, \cdots, j_q}\alpha_{i_1, \cdots, i_p}(x)\beta_{j_1, \cdots, j_q}(x)(dx^{i_1}\wedge \cdots \wedge dx^{i_p})\wedge (dx^{j_1}\wedge \cdots \wedge dx^{j_q}), \\
    &\alpha_p \wedge \beta_q = (-1)^{pq}\beta_q \wedge \alpha_p. \\
\end{align}
このことから, 集合$\Omega^{*}(M):=\bigoplus_{k}\Omega^{k}(M)$を定めた時, これは各$p\in M$ごとに(あるいは, 各開近傍$U\subset M$ごとに)$\mathbb{R}$上の次数付き反可換代数$\Omega^{*}(U)$を定める事がわかる
\footnote{代数構造は大域的には定義されておらず, 各点の近傍$p\in U\subset $にまでしか拡張できない. 
そのため, 微分形式の代数構造に着目する時は, 今後$\Omega^{k}(U)$と書くように努める. }. \\
\subsubsection{外微分}
$d: \Omega^{k}(U)\to \Omega^{k+1}(U)$を, 
\begin{align}
    d\omega 
    = \sum_{h_1\cdots h_p}\sum_{i} \frac{\partial a_{h_1\cdots h_p}}{\partial x_i}dx^{i}\wedge dx^{h_1}\wedge \cdots \wedge dx^{h_p}
\end{align}
\subsubsection{Stokesの定理}
\begin{align}
    \int_{M}d\omega = \int_{\partial M}\omega
\end{align}
\subsubsection{de Rhamコホモロジー}
コホモロジー群というのはホモロジー群
\footnote{位相空間$M$のホモロジー群とは, 簡単に言えば「"境界を取ると消えるもの"と"何らかの境界になっているもの"との間の差を測る指標」であり, 
これは$M$の位相構造に直接的に依存する. 
例えば, ユークリッド空間$\mathbb{R}^n$のような「普通の」空間の中では二者の差はなく(ホモロジー群は自明), トーラス$T^2$では二者は区別される. 
ホモロジー群には様々なバリエーションがあるが, 一番直感的に理解しやすいものは, 三角形分割可能な位相空間に対する単体的ホモロジー
($n$-チェインを$n$-単体の形式和として, 境界準同型を「$n$-単体の図形としての境界を(向きを考慮して)取ってくるもの」として構成するホモロジー)だろう. 
詳しくは, \href{https://ja.wikipedia.org/wiki/単体的ホモロジー}{Wikipediaの記事}を参照. }
の双対(co-)であり(チェイン複体およびその間の境界準同型に対して「転置」を考える事, とも言える), 
「コチェイン複体」と呼ばれる対象および「微分」という操作を用いて構成されるアーベル群の列のことを言う. 
標語的には, これは
「"微分すると0になるもの"と"何らかの微分として書けるもの"との間の差を測る指標」として理解される
\footnote{もちろん(?), コホモロジー理論には公理的な定式化があり, 
"Eilenberg-Steenrod公理"と呼ばれるコホモロジー理論を特徴付ける一連の公理系が存在する: \\
位相空間の組$(X, A\subseteq X)$の圏からアーベル群の圏への反変関手の組$h^{i}$であって, \\
「次元公理(一点空間$X=*$に対して, $h^i(*)=0(i\neq 0), \mathbb{Z}(i=0)$)」\\
「切除公理($U\subseteq A \subseteq X$に対して, 同型$h^{n}(X\backslash U, A\backslash U)\simeq h^n(X, A)$が成り立つ)」\\
「ホモトピー公理(位相空間の組の間のホモトピックな2つの連続写像$f, g$から誘導される群準同型$h^i(f), h^i(g)$は同じ)」\\
「完全性公理(長完全列: $\cdots\to h^i(X, A)\to h^i(X)\to h^i(A)\stackrel{d}{\to}h^{i+1}(X, A)\to\cdots$が成立する)」\\
「加法性公理($(X, A)=(\sqcup_\alpha X_\alpha, \sqcup_{\alpha}A_\alpha)$に対して, 
包含写像$(X_\alpha, A_\alpha)\hookrightarrow (X, A)$は同型$h^i(X, A)\simeq  \prod_{\alpha}h^{i}(X_\alpha, A_\alpha)$を引き起こす)」\\
の5つの性質を満たすものを, コホモロジー理論という. 
詳細を述べる余裕はないので省略. }
. 
コホモロジー群は, コチェイン複体の構成の仕方に応じて様々な種類があるが, ここでは微分形式から構成される\textbf{de Rhamコホモロジー群}というものについて述べる. \\
\subsubsection{Hodge双対}
時空全体が$d+1$次元の時, 
$\star: \Omega^{p}(U)\to \Omega^{d+1-p}(U)$
\subsection{ゲージ理論}
ゲージ理論とは, 局所変換(時空の各点上の場$\phi(x)$に対する, 座標に依存した異なる変換$g(x)$)
の下で不変なラグランジアンを持つ場の理論である. 
ゲージ不変性を持つラグランジアンには, 物質場の他に物質場と結合したゲージ場と呼ばれるベクトル場が登場する. 
このような特殊な不変性は, (理論に新たな場の存在を要請するため)通常の意味での対称性というよりは, むしろ理論の冗長性として考えるべきである. 
\subsubsection{(連続かつ非可換な)ゲージ理論の概要}
\subsubsection{数学的な定義}
場の理論は, 「時空多様体の各点に場$\phi$が定義されている」という構造をしており, 
特に場$\phi$が各点上でベクトル空間$V$の構造を持つ場合, 
場の配位は「時空多様体$M^{d+1}$を底空間, ベクトル空間$V$をファイバーとするベクトル束」として定義される
(スカラー場の場合はファイバーが1次元スカラーである線束, スピノル場の場合はさらにスピン構造
\footnote{向きづけ可能な(変換関数$g_{ij}: U_{i}\cap U_j\to \mathrm{GL}(n)$が$SO(n)$値となる)ベクトル束$(E, \pi)$について, 
ファイバー$F_x\simeq V$に内積が定義されているとする. 
この時, 各点で$F_x$の正規直交標構(順序付けされた基底の組)を考える事で
$M^{d+1}$上の標構束$P_{SO(E)}$を与える事ができる. 
この標構束に対して, 主束$P_{Spin}(E)$への持ち上げが存在する(すなわち, 束写像$\varphi: P_{Spin}(E)\to P_{SO(E)}$であって, 
任意の$p\in P_{Spin}(E), g\in \mathrm{Spin}(n)$および「二重被覆」を表す準同型$\rho: \mathrm{Spin}(n)\to \mathrm{SO}(n)$
に対して$\varphi(pg) = \varphi(p)\rho(g)$を満たすようなものが存在する)時, 
ベクトル束$(E, \pi)$はスピン構造を持つという. }%
と呼ばれる構造を必要とする). 
この時, 場$\phi$は時空上の各点$x\in M^{d+1}$にベクトル空間の元$\phi(x)$を対応させる対応関係であり, 
これはベクトル束の切断$\phi: U(\subset M^{d+1})\to V$の構造を持つ. \\
 ゲージ理論もまた, 時空多様体$M^{d+1}$上のベクトル束の言葉で定式化される. 
ゲージ群を$G$とする時, 物質場$\phi$は主$G$束$P$に対する同伴ベクトル束$P\times_\phi V$の切断として, 
ゲージ場$A$は主$G$束の$\mathfrak{g}$値接続1-形式
\footnote{ベクトル束の接続とは, ベクトル束$(E, \pi)$および$E$の切断全体の集合$\Gamma(E)$に対して定義される汎函数
$\nabla : \mathfrak{X}(M)\times \Gamma(E)\to \Gamma(E)$
であり, $\mathfrak{X}(M)$, $\Gamma(E)$に関する線形性および$\Gamma(E)$に関するライプニッツ則
$\nabla_{X}(fs) = X(f)s + f\nabla_X s$を満たすもののことを言う. 
得られる$\nabla_X s$を「$\nabla$によって定められる$s$の$X$方向の共変微分」という. \\
 ベクトル場$X\in \mathfrak{X}(M)$を明示しない接続の別の定式化がある. 
$p\in M$を指定した時, 全空間$E$に値を取る写像
$$\nabla s|_p : X_p\in T_{p}M \mapsto \nabla_{X_p}s|_p \in E$$
を定義すると, この$\nabla s|_p$は$T^{*}_p M\otimes E$の元とみなせる. 
そこで, $M$上の各点$p$ごとに$T^{*}_p M$の元$\nabla s|_p$を対応させるような切断
$$\nabla s: M \to T^{*}M\otimes E$$を考えると, 
$\nabla$は切断$s: M\to E(\in \Gamma(M))$に切断$\nabla_s: M\to T^{*}M\otimes E (\in \Gamma(T^{*}M\otimes E))$を対応させる(線形な)写像として定義できる. \\
%接続形式の定義
 ベクトル束の接続$1$-形式とは, 
}%
$\Gamma(U, \Omega^1(U)\times \mathfrak{g})$として与えられる. 
ここで, 主$G$束, および主束に同伴するベクトル束の定義は, それぞれ以下の通り. 
\begin{definition}[主$G$束]
    $G$を位相群(群演算が$G$上の連続写像として書けるような群), $P$を
    連続な右作用$\rho: P\times G \to G$が定まっているような位相空間とする. 
    この$P$を同値関係
    \begin{align}
        x\sim x' \Leftrightarrow \exists g\in G~s.t. ~ x'=xg
    \end{align}
    で割ったものを$B=P/\sim $とする. 
    この右作用が自由(つまり, $\forall x\in P, 
    \forall g, h\in G$に対して$xg=xh$ならば$g=h$となる)ならば, 
    標準的な射影$\pi: P\to B$は主$G$束(principal $G$-bundle)であるという. \\
     主$G$束は, しばしば短完全系列
    \begin{align}
        0\incl G\to P\stackrel{\pi}{\to} B\to 0
    \label{pGbundle}
    \end{align}
    によって表される. 
\end{definition}
\begin{definition}[同伴ベクトル束]
    $\rho$
\end{definition}
また, リー群$G$の共役表現およびリー代数$\mathfrak{g}$の随伴表現は, それぞれ次で定義される表現である. 
\begin{definition}[共役表現]
    $G$
\end{definition}
\begin{definition}[随伴表現]
    $\mathfrak{g}$
\end{definition}
%definitionを述べる
 ゲージ理論において重要な量として, ホロノミーと呼ばれるものがある. 
\begin{definition}[水平持ち上げ]
    $\rho$
\end{definition}
\begin{definition}[ホロノミー]
    $\rho$
\end{definition}
\subsubsection{局所対称性: 背景ゲージ場との結合}
ゲージ理論においては, 力学的自由度としての物質場の他に, ゲージ自由度(冗長性)を記述する背景ゲージ場が存在し, 
物質場のカレントがこのゲージ場と結合してそれぞれが変換則に従うことで, 全体として局所ゲージ不変な作用を構成している. 
このようにして, 大域的ゲージ対称性(あるいは, そのカレント)から局所ゲージ不変な理論の作用を与える手続きを, 局所ゲージ化と呼ぶ. \\
 背景ゲージ場は, 力学変数ではなくあらかじめ時空に与えられた古典的な場であり, 
大域的対称性のみを考える上では理論に登場しない(見えない). 
これは, 局所ゲージ不変な場の理論を定義するために必要な「環境」あるいは「舞台」のようなものと思えば良い(のか?). \\
 理論にゲージ不変性がある時, ゲージ群$G$に対する大域的変換の下での($0$次)対称性から, 保存カレントが存在する: 
\begin{align}
    \delta S = \int \partial_\mu \epsilon(x)j^\mu (x)d^dx, ~ 
    \mathrm{for}~ \phi(x)\mapsto \phi(x)+ \epsilon(x)\delta \phi(x). 
\end{align}
この保存カレント$j^{\mu}$に対して1-form $j=j_{\mu}dx^{\mu}$を考えると, 
そのHodge双対は閉形式となる: $d\star j=0$. 
以下, この閉$d$-formのことを保存カレント$\star j$と呼ぶことにする. \\\\
 保存カレントが存在する時, それと背景ゲージ場との結合は, 作用に次の項を加えることで行われる: 
\begin{align}
    S + = 2\pi i\int A_{\mu}(x)j^\mu(x)d^d x = 2\pi i\int A\wedge \star j (=: S_{gauge}). 
\end{align}
係数$2\pi i$は単なる規格化の因子. 
カレントが$d$-formの時, 作用全体は$(d+1)$-formの$(d+1)$次元時空にわたる積分の形になるべきなので, 結合する背景ゲージ場は$1$-formとなる
(higher-form symmetryの場合, より高次の背景ゲージ場との結合を考えることになる). \\
 暫くは簡単のため$U(1)$の$0$次対称性を考える. いま, 場$\psi$の微小な局所的内部変換$\psi(x)\mapsto \psi(x) + i\epsilon(x)q\psi(x)$に対して, 作用の変化分は
(ここの計算は, 内部対称性に対するNoetherカレントの導出と同じ)
\begin{align}
    \delta S = \int d^{d+1}x \frac{\partial \mathcal{L}}{\partial(\partial_\mu \psi)}iq\psi(x)\partial_\mu\epsilon(x)
    =i\int d^{d+1}x j^\mu(x)\partial_\mu \epsilon(x)
    =i\int d^{d+1}x \partial_\mu j^{\mu}(x) \epsilon(x)
    =i\int d\star j^{d} \epsilon
%\label{}
\end{align}
となる(最後の等式は単に微分形式で書き直しただけ). 
カレントと結合する背景ゲージ場がある時は, 結合項が運動方程式を変えるために局所変換のもとでカレントは保存せず, 
物質場のみの変換を考える限りでは作用は不変でなくなる. 
しかし, 
背景ゲージ場のゲージ変換$A\to A + \frac{1}{2\pi}d\epsilon$を同時に行えば, 
全体の作用$S_{tot} = S + S_{gauge}$は
\begin{align}
    \delta S_{tot} = i\int d\star j^{d} \epsilon  +  2\pi i\int \frac{1}{2\pi}d\epsilon \wedge \star j^d
    =i\int d\star j^{d} \epsilon + \int d(\epsilon \wedge \star j^d) - \int \epsilon d \star j^{d}
    =\int d(\epsilon \wedge \star j^d) = 0
\end{align}
となり不変である. 
ゆえ, 背景ゲージ場と結合させた理論は, 変換
$\psi \to \psi + i\epsilon q \psi, ~ A\to A + \frac{1}{2\pi}d\epsilon$($\epsilon: 0$-form, not closed
\footnote{時空全体が連結な場合, closed $0$-form全体の集合は(局所)定数関数である
(このことは, 連結な微分可能多様体$M$に対して, 
その$0$次のde Rhamコホモロジー群が$H^{0}_{\mathrm{dR}}(M)\simeq \re$であるとも言える). 
特に, 平坦な時空(あるいはそれとホモトピー同値な時空)の場合, 開被覆は$U=M$の1枚のみなので, closed $0$-formは大域的に定数な関数に他ならない. 
そのような$\epsilon$による変換は大域的変換であり, Noetherカレントの定義から, この変換の下で作用は自明に不変. \\
(メモ)時空が平坦(up to homotopy)でない場合, 異なる開近傍ごとに異なる変換を施すことも「closed $0$-formによる変換」として許されるので, 
開近傍の交叉の上で滑らかでない内部変換を考えることができ, もう少し面白いことが言える
(ちなみに, このようなことを球面上の物質場がない=pureな$U(1)$ゲージ理論に対して考えると, $dB\neq 0$ at $x\in U_{N}\cap U_{S}$, すなわち$\mathrm{div} \bm{B}\neq 0$
となりDirac monopoleと呼ばれるトポロジカルに非自明な配位が出現する). 
}
)
に対して不変であることが分かる. 
\subsubsection{'t Hooftアノマリー}
%"symmetry defect operators implement a background gauge transformation along its world volume"
 ある理論の作用に対して, カレントと背景ゲージ場とを結合させると, カレントの保存則は
「背景ゲージ場$A$に対するゲージ変換のもとで分配関数が不変となること」に言い換えられる. 
いま, 背景ゲージ場と結合させた物質場の作用を$S[\psi; A]$のように書く時, 理論の分配関数は
\begin{align}
    Z[A]=\pint{\psi}e^{-S[\psi; A]}
\label{partitionfunc}
\end{align}
と書ける. 背景ゲージ場はあくまで「背景」場であり経路積分の対象としないため, この分配関数は背景場の汎函数になる. 
ここで, この分配関数に対して, 背景場の変換$A\to A + d\lambda$を行うと, 
\begin{align}
    Z[A+d\lambda]&=\pint{\psi}e^{-S[\psi; A+d\lambda]}\\
    &=\pint{\psi}e^{-S[g^{-1}\psi; A]} ~~(作用の不変性: S[\psi, A]=S[g\psi, A+\delta]~\mathrm{for}~\exists g\in U(1). g=e^{iq\lambda}. )\\
    &=\pint{(g\psi)}e^{-S[\psi; A]} ~~(経路積分の変数変換)
\label{partitionfunc}
\end{align}
となる. この経路積分測度$\mathcal{D}\psi = \prod_{x_i}d\psi(x_j)$
\footnote{この経路積分測度の書き方は非常にインフォーマルであり, $j$は「時空格子の各格子点のラベル」だと思えばとりあえずOK. 
厳密には時空格子の格子間隔$a\to 0$の極限を取るべきだが, そのような極限の下で発散しない測度を構成できるか? という疑問が残る. }が
$U(1)$変換(あるいは, 一般のゲージ群$G$の元による内部変換)の下で不変にならない時, すなわち何らかの非自明な位相
$\mathcal{D}(g\psi) = e^{i\int \alpha[A, \lambda]}\mathcal{D}\psi$
を獲得してしまう時
\footnote{このように, ある大域的対称性$G$の局所ゲージ化のもとで分配関数のゲージ不変性が破れるか否かを, 
物質場$\psi$(より一般の言い方をすれば, 理論における力学的自由度)の経路積分測度$\mathcal{D}\psi$の変換則, すなわち変数変換のJacobianを調べることによって判定する方法を, 
考案者の名前を取って「藤川の方法」という. 
経路積分測度は「時空上のあらゆる点において, あらゆる可能な場の配位について考慮せよ(足し上げよ)」というものであるから, 
これは変換群$G$のみならず系の力学そのものに依存して決まる. 
ゆえ, 分配関数の不変性の破れを調べる問題は, 「どのような大域的対称性のもとで, どのような力学(あるいは作用)を考えるか」に依る問題である. \\
 なお, ゲージ群$G$の選択によっては, どのような物理的な(=背景時空の対称性を尊重した)作用に対しても't Hooftアノマリーが出現しないということが知られている. 
この点は, 九後「ゲージ場の量子論(II)」の9章-2にも言及がある. 
}
, 分配関数は背景場のグローバルな変換$A\to A+ d\lambda$の下で不変にならない. 
これは, 理論の大域的対称性を局所ゲージ化する際に, 分配関数の不変性を保てなくなってしまうという事態を意味する. 
この「大域的対称な理論を局所ゲージ化すると分配関数が不変でなくなること」を, 理論に\textbf{ゲージアノマリー}(摂動アノマリー, 't Hooftアノマリー)があるといい
\footnote{厳密にはこれらは区別されるもの......なんだっけ?}, 
この不変性の破れを表す位相$e^{i\int \alpha[A, \lambda]}=Z[A+\lambda]/Z[A]$の事を\textbf{'t Hooftアノマリー}と呼ぶ
\footnote{この't Hooftアノマリー(位相の変化分)は, 背景場の有効作用$\Gamma[A]=\log Z[A]$の変化分$\delta_\lambda \Gamma[A]$との間に
\begin{align}
    e^{i\int \alpha[A, \lambda]} = 1-i\delta_\lambda \Gamma[A] + (\lambda の高次項)
\end{align}
という関係がある. }. 
%アノマリー流入仮説について
 't Hooftアノマリーのある理論の分配関数から, ゲージ不変な分配関数を与える処方が存在する. 
 暫くは, $D=d+1$として, $D$次元の$U(1)$ 't Hooftアノマリーがある場の理論を考えることとする. 
\subsection{共形場理論(CFT)}
Non-invertible symmetryの重要な例の多くは, $1+1$次元のRational CFT (RCFT)において発見されてきました
(例えば, \cite{EV}など). 
ここでは, 共形場理論の重要な概念についてなるべく説明を試みます. 
\subsubsection{共形対称性とは}
共形場理論では, 共形変換とよばれる次の変換について考える: 
\begin{definition}{共形変換}
    計量を持つ多様体(物理的には, 時空多様体)$(M, g)$, $(N, g)$の間の局所微分同相$\varphi: U\to V$が, 
    ある$C^\infty$関数$\Omega: U\to \re$を用いて
    \begin{align}
        \varphi^{*}g' = \Omega^2 g
    \end{align}
    と書ける時, $\varphi$を共形変換と呼ぶ. 
    ここで, $\varphi^*$は$\varphi$による計量の引き戻しである. \\
     $U$および$V$を$\real{d}$のもとで局所座標表示すると, $\varphi$による変換は座標変換
    $(x^1, \cdots, x^n)\mapsto \varphi(x^1, \cdots, x^n)=(\varphi(x^1), \cdots, \varphi(x^n))$を引き起こす. 
    この時, 共形変換の定義式(計量の引き戻しに関する式)は
    \begin{align}
        (\varphi^{*}g')_{\mu\nu}(x)=g^{'}_{\rho\sigma}(\varphi(x))\frac{\partial \varphi^\rho}{\partial x^\mu}\frac{\partial\varphi^\sigma}{\partial x^\nu}
        =\Omega^2 g_{\mu\nu}(x)
    \end{align}
    と書ける. 
\end{definition}
特別な場合として時空を$1+1=2$次元に取り, $z=x+i\tau$として$\real{2}$を$\mathbb{C}$とみなせば, 
共形変換は局所的に可逆な正則変換(ないし反正則変換)として書ける. \\
 大域的な共形変換(時空上の全ての点に同じ変換を施す操作)は, 回転, スケール変換, 平行, およびその任意の合成で書ける. Riemann球面$\mathbb{C}\cup {\infty}$($\{\infty\}$: 無限遠点)では, これは
「メビウス変換」
\begin{align}
    \varphi: z\mapsto \frac{az + b}{cz + d}, ~ a, b, c, d\in \mathbb{C}, ad-bc=1
    \label{Mobius}
\end{align}
として書ける(この変換全体は行列群$\mathrm{SL}(2, \mathbb{C})$をなす). 
\subsubsection{プライマリ場と相関関数}
以下, 共形変換$z\mapsto w$の元で「良い振る舞いをする」場の演算子を考えたい. 
\begin{definition}{プライマリー場, 共形ウェイト}
    (局所変換も含めた一般的な)共形変換$z\mapsto w, \bar{z}\mapsto \bar{w}$の下で
    \begin{align}
        \phi(z, \bar{z}) = \phi'(w, \bar{w})\ler{\frac{dw}{dz}}^h \ler{\frac{d\bar{w}}{d\bar{z}}}^{\bar{h}}
    \end{align}
    のように変換する場を\textbf{プライマリ場}といい, この時の指数$(h, \bar{h})$を\textbf{共形ウェイト}という. 
\end{definition}
このような場は, スケール変換$z\mapsto \lambda z(\lambda\in \re)$および回転$z\mapsto e^{i\theta}z$に対してそれぞれ
\begin{align}
    \phi(z, \bar(z))= \phi'(\lambda z, \lambda\bar{z})\lambda^{h+\bar{h}}, ~~
    \phi(z, \bar(z))= \phi'(e^{i\theta}z, e^{-i\theta}\bar{z})e^{i\theta(h-\bar{h})}
\end{align}
と変換する. $h+\bar{h}=: \Delta$はスケーリング次元, $h-\bar{h}=: s$はスピンと呼ばれる. 
なお, 大域的共形変換のみに対してこのような変換則を満たすものは, 「準」プライマリ場と呼ぶ. 
準プライマリ場の重要な例として, 後述するエネルギー運動量テンソルなどが挙げられる. \\
 プライマリ場の無限小共形変換$z\mapsto z + \epsilon(z)$を考えると, (計算略)
\begin{align}
    \phi (z, \bar{z})\mapsto 
    \phi (z, \bar{z}) + \slr{\epsilon(z)\frac{\partial}{\partial z}\phi(z, \bar{z})-h\epsilon(z)\phi(z, \bar{z})}
    +\slr{\bar{\epsilon}(\bar{z})\frac{\partial}{\partial \bar{z}}\phi(z, \bar{z})-\bar{h}\bar{\epsilon}(\bar{z}\phi(z, \bar{z}))}
    +\mathcal{O}(\epsilon^2)
\end{align}
となる. すなわち, プライマリ場の無限小変換則は, 
正則部分$\phi_h(z)$と反正則部分$\phi'_{\bar{h}}(\bar{z})$に完全に分けて扱うことができる. \\
\\
 プライマリ場の重要な点として, 共形対称性から相関関数の形が強い制約を受けるというものがある. 
対称性の帰結として, 大域的な共形変換のもとで相関関数は不変に保たれる: 
\begin{align}
    \sum_{i=1}^{N}\lr{\phi_1(z_1, \bar{z}_1)\cdots \delta_\epsilon\phi_i(z_i, \bar{z}_i)\cdots \phi_N(z_N, \bar{z}_N)}
    \label{Csym}. 
\end{align}
ここで, \eqref{Mobius}で$a-1, b, c, d-1$を無限小に取る(要するに, 恒等変換$a=d=1, b=c=0$にきわめて近い変換を考える)と, 
\begin{align}
    z\mapsto z + b + (a-1+d-1)z -cz^2 + (高次項)
\end{align}
となるから, 無限小の大域的共形変換は$1, z, z^2$とその線型結合によって書ける. 
これを踏まえて, 大域的共形変換$z\mapsto z + \epsilon(z)$を$\epsilon=1, z, z^2$に対して施せば, \eqref{Csym}からそれぞれ
\begin{align}
    \sum_{i}\lr{\phi_1(z_1)\cdots \frac{\partial}{\partial z_i}\phi_i(z_i)\cdots \phi_N(z_N)}=0, 
\end{align}
\begin{align}
    \sum_{i}\lr{\phi_1(z_1)\cdots \ler{h_i + z_i\frac{\partial}{\partial z_i}}\phi_i(z_i)\cdots \phi_N(z_N)}=0, 
\end{align}
\begin{align}
    \sum_{i}\lr{\phi_1(z_1)\cdots \ler{2h_i z_i + z_i^2\frac{\partial}{\partial z_i}\phi_i(z_i)}\cdots \phi_N(z_N)}=0
\end{align}
となる. ただし, 
プライマリ場の性質上, 無限小変換は正則部分のみ考えれば良い(反正則部分についてはparallelである)ことを考慮して, 
$z$依存性のみ表記した(以後も事に応じてプライマリ場についてはそうする). 
これらの式, および
並進対称性から$n$点相関関数の$z_i$依存性は必ず$z_i-z_J$の形で入ることを用いれば, 1点, 2点, 3点関数の形が(計算略)
\begin{align}
    \lr{\phi(z)} = \left\{
    \begin{array}{ll}
    \mathrm{const. } & (h=0)\\
    0 & (h\neq 0)
    \end{array}
    \right.
\end{align}
\begin{equation}
    \lr{\phi_1(z_1, \bar{z_1})\phi_2(z_2, \bar{z}_2)} = \left\{
    \begin{array}{ll}
    \frac{\mathrm{const.}}{(z_1-z_2)^{2h_1}(\bar{z}_1-\bar{z}_2)^{2h_2}} & (h_1=h_2, \bar{h}_1 = \bar{h}_2)\\
    0 & (\mathrm{otherwise})
    \end{array}
    \right.
\end{equation}
\begin{align}
    &\lr{\phi_1(z_1, \bar{z_1})\phi_2(z_2, \bar{z}_2)\phi_3(z_3, \bar{z}_3)}\\
    &= \frac{C_{123}}{(z_1-z_2)^{h_1 + h_2 - h_3}(z_2-z_3)^{h_2 + h_3 - h_1}(z_3-z_1)^{h_3+h_1-h_2}
    (\bar{z}_1-\bar{z}_2)^{\bar{h}_1+\bar{h}_2-\bar{h_3}}(\bar{z}_2-\bar{z}_3)^{h_2 + h_3 - h_1}(\bar{z}_3-\bar{z}_1)^{h_3+ h_1-h_2}}\\
    &=: \frac{C_{123}}{z_12^{h_12}z_{23}^{h_23}z_{31}^{h_31}\bar{z}_{12}^{\bar{h}_12}\bar{z}_{23}^{\bar{h}_23}\bar{z}_{31}^{\bar{h}_{31}}}
\end{align}
とまで決定できる($z_{ij}=z_i-z_j, h_{ab}=h_a + h_b - h_c~~(i,j,a,b,c\in \{1,2,3\})$と略記). 
未知の係数は別の方法(演算子積展開)で決める. \\
 4点関数については, 後述の通り, 演算子積展開よって3点関数に帰着することで形を制限できる(座標依存性を完全に決定することはできない). 
\subsubsection{エネルギー運動量テンソル}
\textbf{エネルギー運動量テンソル}は, 通常は時空の無限小並進$x\mu\to x^\mu-\epsilon^\mu$
\footnote{より正確には, 時空の無限小並進と場の相対的不変性から誘導される場の無限小変換}
に対する保存カレント
\begin{align}
    T^{\mu}_\nu = \frac{\partial\mathcal{L}}{\partial(\partial_\mu \phi)}\partial_\nu \phi - \delta^\mu_\nu \mathcal{L}
\end{align}
として定義されるが, CFTの文脈では計量テンソルの無限小変換
$g_{\mu\nu}\to g _{\mu\nu} + \delta g_{\mu\nu}$
に対する作用の変分によって定義する: 
\begin{align}
    \delta S = \int d^d x \frac{\delta S}{\delta g_{\mu\nu}}(x)\delta_{\mu\nu}(x)
    =: \frac{1}{4\pi}\int d^d x \sqrt{|\det g|}\ler{\frac{4\pi}{\sqrt{|\det g|}}T^{\mu\nu}}\delta g_{\mu\nu}~. 
\end{align}
係数$\sqrt{|\det g|}$は曲がった時空の場合でも扱いやすくするためにつけている. 
この座標表示のもとで, $T^{\mu\nu}$はトレースが$0$の対称テンソルである
\footnote{ただし, 計量の変換によって相関関数の定義式における経路積分測度が不変であるとした. 
この不変性が破れる時, $T^{\mu\nu}$がトレースレスにならないことがある(トレースアノマリー). }. \\
平坦な$1+1$次元時空で複素座標を用いる($z=x_1 + ix_2$)と, 各成分は
\begin{align}
    T_{zz} &= \frac{1}{2}(T_11 - iT_{12})\\
    T_{\bar{z}\bar{z}} &= \frac{1}{2}(T_11+iT_{12})\\
    T_{z\bar{z}}&=T_{\bar{z}z}=0\\
\end{align}
となる(最後の2つは実座標におけるトレースレスネス). 
また, 保存則は
\begin{align}
    \frac{\partial}{\partial \bar{z}}T_{zz}=0, ~\frac{\partial}{\partial z}T_{\bar{z}\bar{z}}=0
\end{align}
と書ける. すなわち, 複素座標系では$T_{zz}$は正則な, $T_{\bar{z}\bar{z}}$は反正則な関数となる: 
$T_{zz}=T(z), ~ T_{\bar{z}\bar{z}}=\bar{T}(\bar{z})$. \\
この$T(z)$を用いれば, 
一般の無限小微分同相写像(局所変換)$z\mapsto z + \epsilon(z)$に対する相関関数の不変性からの帰結として, 
\textbf{共形Ward-高橋恒等式}
\begin{align}
    \label{CWT}
    \sum_{i=1}^{N}\lr{\phi_1(z_1, \bar{z}_1)\cdot \delta_\epsilon\phi_i(z_i, \bar{z}_i)\cdots \phi_N(z_N, \bar{z}_N)}
    +\frac{1}{\pi}\int_{M} d^2 z \partial_z \epsilon^z \langle T(z) \phi_1(z_1, \bar{z}_1)\cdots \phi_N(z_N, \bar{z}_N)\rangle
    =0
\end{align}
を得る. 第2項は時空全体での積分であるが, これをCauchyの積分定理を用いて点$z_i\in M$を囲む円周上の積分に取り直すと, 
\begin{align}
    \delta_\epsilon \phi(z_i, \bar{z}_i) = \frac{1}{2\pi}\oint_{z_i}dz \epsilon(z)T(z)\phi_i(z_i, \bar{z}_i)
\end{align}
となる. 
\subsubsection{演算子積(OPE)}
一般に, QFTにおいて, 2つの局所演算子が時空上の十分近い点$z, w$($|z-w|$: sufficiently small
\footnote{何をもってsufficientとするかは, 考えている理論のスケールによる問題である. 
例えば, ものすごく遠くから系を見る(格子間隔を小さいとみなす$\leftrightarrow$低エネルギースケールで考える)場合, 
この展開はそのスケールにおいてより厳密になる. }
)に存在する時, それらを1つの点での局所演算子によって展開できる. 
%プライマリ場の場合, 
この時の展開係数は一般に$z$と$w$の関数となる: 
\begin{align}
    \phi_i(z) \phi_j(w)=\sum_{k\in }
\end{align}
OPEを利用して, プライマリ場の4点関数について調べ良い. 
まず, 大域的共形不変性から, 4点関数は最も一般には次のように書けることがわかる: 
\begin{align}
    \lr{\phi_1(z_1, \bar{z}_1)\phi_2(z_2, \bar{z}_2)\phi_3(z_3, \bar{z}_3)\phi_4(z_4, \bar{z}_4)}
    =F\ler{\frac{z_{12}z_{34}}{z_{13}z_{24}}, \bar{\frac{z_{12}z_{34}}{z_{13}z_{24}}}}
    \prod_{j<l}^{4}z_{jl}^{(h/3 - h_j - h_l)}z_{jl}^{\bar{h}/3 - \bar{h}_j - \bar{h}_l}~. 
\end{align}
ただし, $h=\sum_{j=1}^4h_j, ~ \bar{h}=\sum_{j=1}^4 \bar{h}_j$. 
未知の係数関数$F$は, 大域的共形不変性によって, 交差比$\frac{z_{12}z_{34}}{z_{13}z_{24}}$のみの関数であることがわかる. 
交差比は4つの座標点のうち3つを任意に移動させても不変であるので, $z=z_3$として
$$(z_1, z_2, z_3, z_4)\mapsto (\infty, 1, z, 0)$$
と移してしまっても良い. この時, 関数
\begin{align}
    G^{21}_{34}\lim_{w, \bar{w}\to \infty}w^{2h_w}\bar{w}^{2\bar{h}_w}\lr{\phi_1(w, \bar{w})\phi_2(1, 1)\phi_3(z, \bar{z})\phi_4(0,0)}
\end{align}
を定義し, これを
\subsubsection{Virasoro代数の構成}
\subsubsection{状態-演算子対応}
\subsubsection{ミニマル模型}
%\subsubsection{Ising CFT: $(p.q)=(3,4)$のミニマル模型}
\begin{thebibliography}{99}
    %\bibitem{Hidaka} \href{}{}
\end{thebibliography}



\end{document}